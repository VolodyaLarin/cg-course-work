\Introduction
Синтез реалистичных изображений в настоящее время переживает смену парадигмы в сторону производства контента в реальном времени. Современные игровые движки способны в режиме реального времени воспроизводить многие реалистичные графические приемы, которые традиционно требовали большого количества времени на вычисления \cite{RONNOW202136}.  
\par
Игровым сценам присущи движения тех или иных объектов, например, бег игрока, полет самолетов и другие. В отличие от снимков движущихся объектов, сделанных фотокамерой, отрисованное изображение будет четким, что лишает зрителя информации о движении объекта. Из-за этого отрисованная анимация объектов может казаться разорванной \cite{Navarro11}.   Данная проблема актуальна, так как рынок видеоигр стремительно развивается, а вычислительной мощности устройств не всегда хватает, чтобы достичь достаточной частоты обновления экрана.
\par
Данную проблему можно решить с помощью добавления на сгенерированное изображение размытия движения. Размытие движения - это эффект, проявляющийся в виде видимых полос, возникающих при движении объекта перед записывающим устройством. Размытие может быть вызвано движением объекта и/или камеры.  Размытие может наблюдаться как на неподвижных фотографиях, так и в последовательностях изображений \cite{Navarro11}.  

% Синтез реалистичных изображений в настоящее время переживает смену парадигмы в сторону производства контента в реальном времени. Современные игровые движки способны в режиме реального времени воспроизводить многие реалистичные графические приемы, которые традиционно требовали большого количества времени на вычисления.
% \par
% Некоторые из этих приемов все еще грубо аппроксимируются, что приводит к появлению видимых артефактов. Для достижения эффекта интерактивности и скрытию этих артифактов можно использовать размытие. Одним из таких эффектов является размытие движения, которое необходимо для представления движущихся объектов. Размытие движения - это распространенный оптический эффект на фотографиях и видео, который возникает, когда положение объектов меняется относительно точки зрения камеры в течение промежутка времени, когда затвор камеры открыт. Если объекты движутся быстро, или интервал между затворами достаточно длинный, то объекты оставляют размытые полосы в направлении движения. Важно воспроизвести этот эффект для синтеза захватывающих и более правдоподобных сцен, имитации определенных моделей камер или достижения художественных эффектов.
% \par
% Размытие движения - это эффект, проявляющийся в виде видимых полос, возникающих при движении
% объекта перед записывающим устройством. Это результат сочетания видимого движения в сцене и
% особенностей средства формирования изображения, которое объединяет свет в течение ограниченного времени съемки.
% \par
% Это относительное движение может быть вызвано движением объекта и движением камеры
% и может наблюдаться как на неподвижных фотографиях, так и в последовательностях изображений.
% В целом, последовательности, содержащие умеренное размытие движения, воспринимаются как
% естественные, в то время как его полное отсутствие приводит к рывкам и потере реалистичности.
% \par
% Размытие движения для синтезированных изображений является активной областью исследований
% с начала 1980-х годов. В отличие от записанных видеоматериалов, которые автоматически
% размывают движения, алгоритмы рендеринга должны явно моделировать его.
% \par
% В данной работе описывается происхождение этого явления и подробно рассматриваются алгоритмические решения, которые были найдены для его моделирования.
% \par
Целью работы является разработка ПО для генерации эффекта размытия движения полигональных моделей. Для достижения поставленной цели необходимо решить следующие задачи:

\begin{itemize}
    \item определить требования для генерации смаза движения;
    \item проанализировать существующие методы генерации смаза движения;
    \item проанализировать существующие методы генерации объемного изображения;
    \item реализовать генерацию объемного изображения с эффектом размытия движения полигональных моделей;
    \item провести сравнение временных и визуальных характеристик алгоритмов генерации смаза.
\end{itemize}



% Проверяем как у нас работают сокращения, обозначения и определения "---
% MAX,
% \Abbrev{MAX}{Maximus ""--- максимальное значение параметра}
% API
% \Abbrev{API}{application programming interface ""--- внешний интерфейс взаимодействия с приложением}
% с обратным прокси.
% \Define{Обратный прокси}{тип прокси-сервера, который ретранслирует}



