\Introduction
Синтез реалистичных изображений в настоящее время переживает смену парадигмы в сторону производства контента в реальном времени. Современные игровые движки способны в режиме реального времени воспроизводить многие реалистичные графические приемы, которые традиционно требовали большого количества времени на вычисления \cite{RONNOW202136}.  
\par
Игровым сценам присущи движения тех или иных объектов, например, бег игрока, полет самолетов и другие. В отличие от снимков движущихся объектов, сделанных фотокамерой, отрисованное изображение будет четким, что лишает зрителя информации о движении объекта. Из-за этого отрисованная анимация объектов может казаться разорванной \cite{Navarro11}.   Данная проблема актуальна, так как рынок видеоигр стремительно развивается, а вычислительной мощности устройств не всегда хватает, чтобы достичь достаточной частоты обновления экрана.
\par
Данную проблему можно решить с помощью добавления на сгенерированное изображение размытия движения. Размытие движения - это эффект, проявляющийся в виде видимых полос, возникающих при движении объекта перед записывающим устройством. Размытие может быть вызвано движением объекта и/или камеры.  Размытие может наблюдаться как на неподвижных фотографиях, так и в последовательностях изображений \cite{Navarro11}.  

Целью работы является проектирование ПО для генерации эффекта размытия движения полигональных моделей. Для достижения поставленной цели необходимо решить следующие задачи:

\begin{itemize}
    \item определить требования для генерации смаза движения;
    \item проанализировать существующие методы генерации смаза движения;
    \item проанализировать существующие методы генерации объемного изображения;
    \item спроектировать генерацию объемного изображения с эффектом размытия движения полигональных моделей;
\end{itemize}



% Проверяем как у нас работают сокращения, обозначения и определения "---
% MAX,
% \Abbrev{MAX}{Maximus ""--- максимальное значение параметра}
% API
% \Abbrev{API}{application programming interface ""--- внешний интерфейс взаимодействия с приложением}
% с обратным прокси.
% \Define{Обратный прокси}{тип прокси-сервера, который ретранслирует}



