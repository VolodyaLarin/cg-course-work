\chapter{Аналитический раздел}
\label{cha:analysis}
В данном разделе приведены основные сведения о размытие движения, о методах и алгоритмах его программной генерации.

\section{Природа размытия движения}

\subsection{Причины появления смаза}

Причины возникновения размытия изображений при съёмке камерой связаны со способом захвата изображения. Свет попадает на светочувствительную матрицу устройства. Затвор и диафрагма влияют на количество света, попавшего на матрицу. Затвор ограничивает время, в течение которого свет попадает на матрицу и формируется итоговое изображение. Данное время называется выдержкой.
\par
После открытия затвора матрица начинает аккумулировать весь свет, попавший на неё. Так как во время движения объекта отраженные лучи света меняют своё положения, то во время выдержки все проекции объекта на плоскость матрицы будут запечатлены на итоговом изображении.
\par
Данный процесс выдержки может быть формализован с помощью уравнения \eqref{F:F2_1_1} . Во время съемки сцены,  представляет собой содержимое плоскости изображения. Захваченный свет является результатом аккумулирования входящего излучения $L$ в течение времени выдержки $\Delta T$. Функция $f$ моделирует влияние оптики, затвора, диафрагмы, светочувствительности матрицы.
\par
\begin{equation}
    I = \int_{\Delta T} f(t) \cdot L(t) dt
    \label{F:F2_1_1}
\end{equation}
\par
Уравнение \eqref{F:F2_1_1} дает представления о характеристиках получаемых изображений. \cite{Navarro11} Можно сказать, что финальное изображения состоит из объединения моментальных снимков. Моментальный снимок - это снимок с временем выдержки, стремящемся к нулю.

\subsection{Описание характеристик смаза}
\label{cha:analysis_charestic}
Для изображения кадра перемещения сферы радиуса, соизмеримого с одним пикселем, необходимо отобразить путь пройденный этой сферой за время выдержки $\Delta T$. Так как любой объект можно представить в виде набора точек, то конечное изображение будет иметь вид следа перемещения объекта за время выдержки.
\par
Для упрощения задачи характеристики смаза, кривую перемещения любого тела можно представить в виде ломанной линии, где каждая точка ломанной есть точка времени начала/окончания выдержки. Такое упрощение возможно, так как время выдержки обычно равно периоду обновления экрана (При частоте обновления кадра 25 Гц выдержка равна 40 мс). Тогда любой путь объекта во время выдержки будет его перемещением. Для единичного пикселя смаз будет представлять собой отрезок, соединяющий точки нахождения пикселя в моменты времени начала и окончания выдержки.
\par
Из выше указанных замечаний можно выдвинуть следующие требования для генерации смазывания движущихся объектов:
\begin{enumerate}
    \item Смаз содержит изображения объекта в моменты времени начала и окончания выдержки;
    \item Смазывание формируется быть вдоль вектора движения объекта;
    \item Продолжительность смаза зависит от скорости движения объекта;
    \item Объекты без движения не подвергаются смазыванию;
    \item Смаз не терпит разрывов;
    \item Интенсивность изображения без движения соизмерима с интенсивностью смазанного изображения.
\end{enumerate}

\section{Методы размытия движения}

\subsection{Размытие движения с помощью накопительного буфера}
\label{cha:analysis_acum}

Одним из первых решений, которое было предложено - это склеивание нескольких отрендеренных изображений. \cite{Haeberli90} Заметим, что интенсивность изображения должна быть соизмерима с начальной. Тогда получим следующую формулу.
\begin{eqndesc}
    \begin{equation}\label{F:F202}
        I(t, \Delta T, N) = \frac{ \sum_{i=0}^{N} { V({t - \frac{i}{N} \cdot \Delta T})}}{N}
    \end{equation}
    \\
    где $t$ "--- время окончания выдержки \\
    $\Delta T$ "--- продолжительность выдержки \\
    $V(t)$ "--- мгновенный снимок в момент времени $t$\\
    $N$ "--- количество мгновенных снимков \\
    $I(t, \Delta T, N)$ "--- изображение с выдержкой $\Delta T$
\end{eqndesc}

Для достижения наивысшей скорости отрисовки изображения при генерации видео ряда необходимо накапливать результаты генерации последних $N$ моментальных снимков в буфере. Постоянность выдержки $\Delta T$ и частоты обновления экрана $fps$  позволяет на каждой итерации отрисовать следующее количество новых кадров  $k = \frac{N}{fps \cdot \Delta T}$, $k \in Z$
\par
Очевидно, что достигается максимальная скорость работы алгоритма при $k = 1$, т.е. когда $fps = \frac{N}{\Delta T}$


\begin{table}[ht]
    \caption{Зависимость частоты кадров от качества и продолжительности размытия движения}
    \begin{tabular}{|c|r|r|r|r|}
        \hline
        $\Delta T$, c & $N =8$ & $=16$ & $=32$ & $=64$ \\
        \hline
        $1$           & 8      & 16    & 32    & 64    \\
        $\frc{1}{2}$  & 16     & 32    & 64    & 128   \\
        $\frc{1}{4}$  & 32     & 64    & 128   & 256   \\
        $\frc{1}{8}$  & 64     & 128   & 256   & 512   \\
        $\frc{1}{16}$ & 128    & 256   & 512   & 1024  \\
        \hline
    \end{tabular}
    \label{tab:tabular}
\end{table}


Чтобы получить изображение с малой выдержкой при большом количестве семплов нужно иметь большую частоту генерации кадров, что становится преградой для реализации размытия в режиме реального времени данным методом.

\subsection{Размытие движения с помощью скорости пикселя}
\label{cha:analysis_pixelblur}


В пункте \ref{cha:analysis_charestic} были сделаны предположения о природе смаза. Если рассматривать каждый пиксель изображения за время выдержки, то можно восстановить путь пройденный каждым пикселем с помощью мгновенной скорости каждого пикселя в момент начала/окончания выдержки.
\par
Для реализации такого размытия можно использовать следующую идею. Цвет каждого пикселя размытого изображения представить, как среднее арифметическое цветов пикселей вдоль вектора направления скорости вычисляемого пикселя. \cite{GpuGems2008}
\par

\begin{eqndesc}
    \begin{equation}\label{F:F2_2_1}
        I(x,y) = \frac{\sum_{i=0}^{N} {V(x + i \cdot v_x, y + i \cdot v_y)}}{N}
    \end{equation}
    \\
    где $x,y$ "--- координаты пикселя \\
    $v$ "--- мгновенная скорость размываемого пикселя  \\
    $V(x,y)$ "--- цвет пикселя мгновенного снимка \\
    $I(x,y)$ "--- цвет размытого пикселя
\end{eqndesc}




\begin{figure}
    \centering
    \begin{minipage}[h]{0.49\linewidth}
        \center{\includegraphics[width=0.5\linewidth]{img/blur_1.jpg} \\ а)}
    \end{minipage}
    \hfill
    \begin{minipage}[h]{0.49\linewidth}
        \center{\includegraphics[width=0.5\linewidth]{img/blur_2.jpg} \\ б)}
    \end{minipage}
    \caption{Сравнение смаза. \cite{jimenez2014next} \\ а) Исходное изображение б) Изображение с размытием по скорости пикселя}
    \label{fig:pixel_blur}
\end{figure}


Данный подход позволяет генерировать размытие по одному снимку. Данный способ намного быстрее предыдущего, но обладает не идеальными графическими параметрами.
\par
Данный метод оставляет резкую границу раздела заднего плана и смазываемого изображения. Это происходит из-за резкого перепада скорости фона и движущегося тела. Также при перекрытие движущегося тела статическим образуются артефакты отрисовки на границе тел. Неподвижное тело не смазывается, но на границе тел скорость резко меняет свое значение, и в соответствии с формулой \eqref{F:F2_2_1} интенсивность пикселя будет составляться и из части изображения неподвижного тела.
\par
Из выше сказанных слов делаем вывод, что данный метод смазывания работает без видимых артефактов  при равномерно-распределенной скорости каждого пикселя, например, при смене ракурса камеры.

\subsection{Размытие движения с помощью максимальной скорости движения участка изображения и буфера глубины}
\label{cha:analysis_mcgiure}

Проблему резких границ размытия было предложено решать с помощью снижения размерности буфера скоростей и анализа соседних участков буфера. Изображение разбивается на участки размера $k \times k$ пикселей, и для каждого участка ищется максимальная скорость. Такое преобразование из буфера скоростей пикселей можно сделать по формуле \eqref{F:F2_3_1}.
\begin{eqndesc}
    \begin{equation}\label{F:F2_3_1}
        TileMax[x,y] = \max_{u \in [o ; k)} {
        \max_{v \in [o ; k)} {v [kx + u; ky + v]}
        }
    \end{equation}
    , где $\max$ "--- максимальный вектор по длине.
\end{eqndesc}

Чтобы проанализировать все пиксели, которые вносят вклад в размытие того или иного пикселя необходимо, чтобы $|v[x,y]| \le k $ \space $\forall x,y $. Следующим шагом предлагается найти доминирующую скорость для соседей участка и самого участка.

\begin{eqndesc}
    \begin{equation}\label{F:F2_3_2}
        NeighborMax[x,y] = \max_{u \in [-1 ; 1]} {
            \max_{v \in [-1 ;1]} {TileMax [x + u; y + v]}
        }
    \end{equation}
\end{eqndesc}

Использование буфера максимальной скорости движения участков, задаваемого уравнение \eqref{F:F2_3_2}, решает проблему резких границ размытия.
\par
В методе, описанном в пункте \ref{cha:analysis_pixelblur} суммировался цвет всех пикселей, даже тех, которые не являлись частью смаза текущего объекта. Чтобы решить данную проблему, было предложено при расчете каждого пикселе $A$ анализировать вклад $\alpha_i$ для каждого соседнего пикселя $B_i$ вдоль  вектора скорости $NeighborMax[\frac{A_x}{k}, \frac{A_y}{k}]$. Морган Макгуайр предложил считать, что пиксель $B_i$ вносит вклад $\alpha_i$ в смаз пикселя $A$ при следующих условиях:
\begin{enumerate}
    \item Условия вклада переднего плана
          \begin{enumerate}
              \item пиксель $B_i$ ближе к наблюдателю чем $A$
              \item пиксель $A$ находится в пределах разброса точек $B_i$;
          \end{enumerate}
    \item Условия вклада фона
          \begin{enumerate}
              \item пиксель $B_i$ дальше от наблюдателя чем $A$
              \item $B_i$ находится в пределах разброса точек $A$
          \end{enumerate}
    \item Условие для единого тела
          \begin{enumerate}
              \item пиксели $A$ и $B_i$ находятся в пределах разброса точек друг друга;
          \end{enumerate}

\end{enumerate}


Разбросом точек пикселя называем окружность с центром в данном пикселе и радиусом скорости данного пикселя.
\par
Было предложено вклад пикселя считать по следующим формулам:
\begin{eqndesc}
    \begin{eqnarray}
        \alpha_i =
        depthCompare(Z[A], Z[B_i]) \cdot cone(B_i, A, v_{B_i}) + \\
        + depthCompare(Z[B_i], Z[A]) \cdot cone(A, B_i, v_A) + \\
        + cylinder(B,A,v_{B_i}) \cdot cylinder(A,V, v_A) \cdot 2
    \end{eqnarray}
    ,где каждое слагаемое соответствует условиям 1-3.
\end{eqndesc}
\par Тогда цвет каждого пикселя можно представить, как
\begin{equation}
    I(A) = \frac{\frac{V[A]}{|v_A|} +
    \sum_{i=1}^N
    {\alpha_i \cdot V[A + v_A \cdot i]}
    }{\frac{1}{|v_A|} + \sum_{i=1}^N  a_i}
\end{equation}

Описание вспомогательных функций:
\begin{itemize}
    \item $clamp(x, L, R) = min(R, max(L, x))$ - ограничение диапазона значений отрезком $[L;R]$
    \item $depthCompare(z_A, z_B) = clamp(1 - \frac{z_A - z_B}{EXTENT}, 0, 1)$ - проверка отдаленности точки $B$ по стравнению с точкой $A$. $EXTENT$ - константа достаточного расстояния между точками. Обычно константа берется от 1 мм до 10 см \cite{McGuire12}
    \item $cone(v, A, B) = clamp(1 - \frac{|A-B|}{|v|}, 0,1)$ - проверка на вхождение в разброс точек
    \item $smoothstep(x) = 3 t^2 - 2t^3$, где $t = clamp(x, 0, 1)$ -  проверка принадлежности краям отрезка [0, 1]
    \item $smoothstep(x, L, R) = smoothstep(\frac{x - L}{R -
                  L})$ - проверка принадлежности краям отрезка [L; R]
    \item $cilinder(v, A, B) = 1 - smoothstep(0.95 |v|, 1.05|v|, |A - B|)$ - проверка на присутствие на границе разброса точек
\end{itemize}

Такой метод выполняет смазывания за три прохода по плоскости изображения и требует в качестве входных данных буферы глубины и скоростей пикселей. Данный метод может применяться для генерации изображение в реальном времени. данный метод может работать с резкими перепадами значений буферов.

\subsection{Сравнение методов размытия движения}

В пунктах \ref{cha:analysis_acum}, \ref{cha:analysis_pixelblur}, \ref{cha:analysis_mcgiure} были приведены основные сведения о методах генерации размытия движения.
\par

Метод размытия движения с помощью накопительного буфера показывает наиболее близкое к природе решение, но имеет большую вычислительную сложность, что делает его не целесообразным при генерации видео-потока в режиме реального времени, но можно использовать для генерации фотоизображений.
\par
Метод размытия движения с помощью скорости пикселя рассчитан на работу с маленькой выдержкой, что позволяет грубо апроксимировать перемещение всех пикселей изображения. Алгоритм работает за один проход по плоскости изображения, что позволяет его использовать для работы в режиме реального времени. Но для размытия движения объекта могут появляться графические артефакты. При размытии движения камеры при статической сцене - артефакты не проявляются.
\par
Метод размытия движения с помощью максимальной скорости движения участка изображения и буфера глубины является модификацией предыдущего метода с устранением проблем резких границ смаза и артефактов смаза статического переднего плана.


\begin{center}
    \begin{longtable}{|p{0.2\textwidth}|p{0.2\textwidth}|p{0.2\textwidth}|p{0.30\textwidth}|}

        \caption{Сравнение методов генерации размытия движения}
        \label{tab:longtable}
        \\ \hline
        Метод размытия движения                                            & с помощью накопительного буфера & с помощью скорости пикселя & с помощью максимальной скорости движения участка изображения и буфера глубины \\
        \hline \endfirsthead
        \subcaption{Продолжение таблицы~\ref{tab:longtable}}
        \\ \hline \endhead
        \hline \subcaption{Продолжение на след. стр.}
        \endfoot
        \hline \endlastfoot
        Генерация изображения с большой выдержкой                          & +                               & -                          & -                                                                             \\
        \hline
        Генерация изображения с малой выдержкой                            & +                               & +                          & +                                                                             \\
        \hline
        Генерация изображения в режиме реального времени выдержкой         & Только при большой выдержке                               & +                          & +                                                                             \\
        \hline
        Генерация смаза движения перекрывающихся объектов разных скоростей & +                               & С видимыми артефактами                          & +                                                                             \\
    \end{longtable}
\end{center}


\section{Удаление невидимых линий и поверхностей}

\subsection{Алгоритм плавающего горизонта}

\subsection{Алгоритм Робертса}

\subsection{Алгоритм Варнока}

\subsection{Алгоритм, использующий список приоритетов}

\subsection{Алгоритм, использующий z-буфер}

\subsection{Алгоритм определения видимых поверхностей путем трассировки лучей}

\subsection{Сравнение алгоритмов удаления невидимых линий и поверхностей}




% Обратите внимание, что включается не ../dia/..., а inc/dia/...
% В Makefile есть соответствующее правило для inc/dia/*.pdf, которое
% берет исходные файлы из ../dia в этом случае.

% \begin{figure}
%   \centering
%   \includegraphics[width=\textwidth]{inc/dia/rpz-idef0}
%   \caption{Рисунок}
%   \label{fig:fig01}
% \end{figure}

% \begin{figure}
%   \centering
%   \includegraphics[height=0.85\textheight]{inc/img/leonardo}
%   \caption{Предполагаемый автопортрет Леонардо да Винчи}
%   \label{fig:leonardo}
% \end{figure}

% В \cite{Pup09} указано, что...

% Кстати, про картинки. Во-первых, для фигур следует использовать \texttt{[ht]}. Если и после этого картинки вставляются <<не по ГОСТ>>, т.е. слишком далеко от места ссылки, "--- значит у вас в РПЗ \textbf{слишком мало текста}! Хотя и ужасный параметр \texttt{!ht} у окружения \texttt{figure} тоже никто не отменял, только при его использовании документ получается страшный, как в ворде, поэтому просьба так не делать по возможности.

% \section{Существующие подходы к созданию всячины}

% Известны следующие подходы...

% \begin{enumerate}
% \item Перечисление с номерами.
% \item Номера первого уровня. Да, ГОСТ требует именно так "--- сначала буквы, на втором уровне "--- цифры.
% Чуть ниже будет вариант <<нормальной>> нумерации и советы по её изменению.
% Да, мне так нравится: на первом уровне выравнивание элементов как у обычных абзацев. Проверим теперь вложенные списки.
% \begin{enumerate}
% \item Номера второго уровня.
% \item Номера второго уровня. Проверяем на длииииной-предлиииииииииинной строке, что получается.... Сойдёт.
% \end{enumerate}
% \item По мнению Лукьяненко, человеческий мозг старается подвести любую проблему к выбору
%   из трех вариантов.
% \item Четвёртый (и последний) элемент списка.
% \end{enumerate}

% Теперь мы покажем, как изменить нумерацию на «нормальную», если вам этого захочется. Пара команд в начале документа поможет нам.

% \renewcommand{\labelenumi}{\arabic{enumi})}
% \renewcommand{\labelenumii}{\asbuk{enumii})}

% \begin{enumerate}
% \item Изменим нумерацию на более привычную...
% \item ... нарушим этим гост.
% \begin{enumerate}
% \item Но, пожалуй, так лучше.
% \end{enumerate}
% \end{enumerate}

% В заключение покажем произвольные маркеры в списках. Для них нужен пакет \textbf{enumerate}.
% \begin{enumerate}[1.]
% \item Маркер с арабской цифрой и с точкой.
% \item Маркер с арабской цифрой и с точкой.
% \begin{enumerate}[I.]
% \item Римская цифра с точкой.
% \item Римская цифра с точкой.
% \end{enumerate}
% \end{enumerate}

% В отчётах могут быть и таблицы "--- см. табл.~\ref{tab:tabular} и~\ref{tab:longtable}.
% Небольшая таблица делается при помощи \Code{tabular} внутри \Code{table} (последний
% полностью аналогичен \Code{figure}, но добавляет другую подпись).

% \begin{table}[ht]
%   \caption{Пример короткой таблицы с коротким названием}
%   \begin{tabular}{|r|c|c|c|l|}
%   \hline
%   Тело      & $F$ & $V$  & $E$ & $F+V-E-2$ \\
%   \hline
%   Тетраэдр  & 4   & 4    & 6   & 0         \\
%   Куб       & 6   & 8    & 12  & 0         \\
%   Октаэдр   & 8   & 6    & 12  & 0         \\
%   Додекаэдр & 20  & 12   & 30  & 0         \\
%   Икосаэдр  & 12  & 20   & 30  & 0         \\
%   \hline
%   Эйлер     & 666 & 9000 & 42  & $+\infty$ \\
%   \hline
%   \end{tabular}
%   \label{tab:tabular}
% \end{table}

% Для больших таблиц следует использовать пакет \Code{longtable}, позволяющий создавать
% таблицы на несколько страниц по ГОСТ.

% Для того, чтобы длинный текст разбивался на много строк в пределах одной ячейки, надо в
% качестве ее формата задавать \texttt{p} и указывать явно ширину: в мм/дюймах
% (\texttt{110mm}), относительно ширины страницы (\texttt{0.22\textbackslash textwidth})
% и~т.п.

% Можно также использовать уменьшенный шрифт "--- но, пожалуйста, тогда уж во \textbf{всей}
% таблице сразу.

% \begin{center}
%   \begin{longtable}{|p{0.40\textwidth}|c|p{0.30\textwidth}|}
%     \caption{Пример длинной таблицы с длинным названием на много длинных-длинных строк}
%     \label{tab:longtable}
%     \\ \hline
%     Вид шума & Громкость, дБ & Комментарий \\
%     \hline \endfirsthead
%     \subcaption{Продолжение таблицы~\ref{tab:longtable}}
%     \\ \hline \endhead
%     \hline \subcaption{Продолжение на след. стр.}
%     \endfoot
%     \hline \endlastfoot
%     Порог слышимости             & 0     &                                                \\
%     \hline
%     Шепот в тихой библиотеке     & 30    &                                                \\
%     Обычный разговор             & 60-70 &                                                \\
%     Звонок телефона              & 80    & \small{Конечно, это было до эпохи мобильников} \\
%     Уличный шум                  & 85    & \small{(внутри машины)}                        \\
%     Гудок поезда                 & 90    &                                                \\
%     Шум электрички               & 95    &                                                \\
%     \hline
%     Порог здоровой нормы         & 90-95 & \small{Длительное пребывание на более
%     громком шуме может привести к ухудшению слуха}                                        \\
%     \hline
%     Мотоцикл                     & 100   &                                                \\
%     Power Mower                  & 107   & \small{(модель бензокосилки)}                  \\
%     Бензопила                    & 110   & \small{(Doom в целом вреден для здоровья)}     \\
%     Рок-концерт                  & 115   &                                                \\
%     \hline
%     Порог боли                   & 125   & \small{feel the pain}                          \\
%     \hline
%     Клепальный молоток           & 125   & \small{(автор сам не знает, что это)}          \\
%     \hline
%     Порог опасности              & 140   & \small{Даже кратковременное пребывание на
%     шуме большего уровня может привести к необратимым последствиям}                       \\
%     \hline
%     Реактивный двигатель         & 140   &                                                \\
%                                  & 180   & \small{Необратимое полное повреждение
%                                  слуховых органов}                                        \\
%     Самый громкий возможный звук & 194   & \small{Интересно, почему?..}                   \\
%   \end{longtable}
% \end{center}

%%% Local Variables:
%%% mode: latex
%%% TeX-master: "rpz"
%%% End:
