%% Преамбула TeX-файла

% 1. Стиль и язык
\documentclass[utf8x,times, 12pt]{G7-32} % Стиль (по умолчанию будет 14pt)

% Остальные стандартные настройки убраны в preamble.inc.tex.
\include{preamble.inc}

% Настройки листингов.
\ifPDFTeX
\include{listings.inc}
\else
\usepackage{local-minted}
\fi

\usepackage{indentfirst}
\usepackage{ulem} % Нормальное нижнее подчеркивание

% Дополнительное окружения для подписей
\usepackage{array}

\usepackage{svg} 

\usepackage{pdflscape}
% \usepackage{lscape}

% Полезные макросы листингов.
% Любимые команды
\newcommand{\Code}[1]{\textbf{#1}}

\newenvironment{signstabular}[1][1]{
	\renewcommand*{\arraystretch}{#1}
	\tabular
}{
	\endtabular
}



\newcommand{\frc}[2]{\raisebox{2pt}{$#1$}\big/\raisebox{-3pt}{$#2$}}    % a/b, a выше, b ниже

\renewcommand{\labelenumi}{\arabic{enumi})}
\renewcommand{\labelenumii}{\asbuk{enumii})}



\usepackage[T2A]{fontenc} % Поддержка русских букв
% \usepackage[utf8]{inputenc} % Кодировка utf8
% \usepackage[english, russian]{babel} % Языки: русский, английский
% \usepackage{pscyr} % Нормальные шрифты

\usepackage{algorithm}
\usepackage{algpseudocode}
\floatname{algorithm}{Псевдокод}

\algrenewcommand\algorithmicwhile{\textbf{До тех пока}}
\algrenewcommand\algorithmicdo{\textbf{выполнить}}
\algrenewcommand\algorithmicrepeat{\textbf{Повторять}}
\algrenewcommand\algorithmicuntil{\textbf{Пока выполняется}}
\algrenewcommand\algorithmicend{\textbf{Конец}}
\algrenewcommand\algorithmicif{\textbf{Если}}
\algrenewcommand\algorithmicelse{\textbf{иначе}}
\algrenewcommand\algorithmicthen{\textbf{тогда}}
\algrenewcommand\algorithmicfor{\textbf{Цикл}}
\algrenewcommand\algorithmicforall{\textbf{Для всех}}
\algrenewcommand\algorithmicfunction{\textbf{Функция}}
\algrenewcommand\algorithmicprocedure{\textbf{Процедура}}
\algrenewcommand\algorithmicloop{\textbf{Зациклить}}
\algrenewcommand\algorithmicrequire{\textbf{Условия:}}
\algrenewcommand\algorithmicensure{\textbf{Обеспечивающие условия:}}
\algrenewcommand\algorithmicreturn{\textbf{Возвратить}}
\algrenewtext{EndWhile}{\textbf{Конец цикла}}
\algrenewtext{EndLoop}{\textbf{Конец зацикливания}}
\algrenewtext{EndFor}{\textbf{Конец цикла}}
\algrenewtext{EndFunction}{\textbf{Конец функции}}
\algrenewtext{EndProcedure}{\textbf{Конец процедуры}}
\algrenewtext{EndIf}{\textbf{Конец условия}}
\algrenewtext{EndFor}{\textbf{Конец цикла}}
\algrenewtext{BeginAlgorithm}{\textbf{Начало алгоритма}}
\algrenewtext{EndAlgorithm}{\textbf{Конец алгоритма}}
\algrenewtext{BeginBlock}{\textbf{Начало блока. }}
\algrenewtext{EndBlock}{\textbf{Конец блока}}
\algrenewtext{ElsIf}{\textbf{иначе если }}

\renewcommand{\thealgorithm}{\thechapter.\arabic{algorithm}}%

\makeatletter
\@addtoreset{algorithm}{chapter}% algorithm counter resets every chapter
\makeatother



% Fix breaking algos per page 
\makeatletter
\newenvironment{breakablealgorithm}
  {% \begin{breakablealgorithm}
   \begin{center}
     \refstepcounter{algorithm}% New algorithm
     \hrule height.8pt depth0pt \kern2pt% \@fs@pre for \@fs@ruled
     \renewcommand{\caption}[2][\relax]{% Make a new \caption
       {\raggedright\textbf{\fname@algorithm~\thealgorithm} ##2\par}%
       \ifx\relax##1\relax % #1 is \relax
         \addcontentsline{loa}{algorithm}{\protect\numberline{\thealgorithm}##2}%
       \else % #1 is not \relax
         \addcontentsline{loa}{algorithm}{\protect\numberline{\thealgorithm}##1}%
       \fi
       \kern2pt\hrule\kern2pt
     }
  }{% \end{breakablealgorithm}
     \kern2pt\hrule\relax% \@fs@post for \@fs@ruled
   \end{center}
  }
\makeatother


\begin{document}

\frontmatter % выключает нумерацию ВСЕГО; здесь начинаются ненумерованные главы: реферат, введение, глоссарий, сокращения и прочее.

\begin{titlepage}
    \thispagestyle{empty}

    \noindent\begin{minipage}{0.05\textwidth}
        \includegraphics[scale=0.3]{img/bmstu.png}
    \end{minipage}
    \hfill
    \begin{minipage}{0.85\textwidth}\raggedleft
        \begin{center}
            \fontsize{12pt}{0.3\baselineskip}\selectfont \textbf{Министерство науки и высшего образования Российской Федерации \\ Федеральное государственное бюджетное образовательное учреждение \\ высшего образования \\ <<Московский государственный технический университет \\ имени Н.Э. Баумана \\ (национальный исследовательский университет)>> \\ (МГТУ им. Н.Э. Баумана)}
        \end{center}
    \end{minipage}

    \begin{center}
        \fontsize{12pt}{0.1\baselineskip}\selectfont
        \noindent\makebox[\linewidth]{\rule{\textwidth}{4pt}} \makebox[\linewidth]{\rule{\textwidth}{1pt}}
    \end{center}

    \begin{flushleft}
        \fontsize{12pt}{0.8\baselineskip}\selectfont

        ФАКУЛЬТЕТ \uline{
            Информатика и системы управления
            \hfill}

        КАФЕДРА \uline{\mbox{\hspace{4mm}}
            Программное обеспечение ЭВМ и информационные технологии
            \hfill}
    \end{flushleft}

    \vfill

    \begin{center}
        \fontsize{20pt}{\baselineskip}\selectfont

        \textbf{ОТЧЕТ ПО ПРОИЗВОДСТВЕННОЙ ПРАКТИКЕ}
    \end{center}



    \vfill
    \begin{table}[h!]
        \fontsize{14pt}{0.7\baselineskip}\selectfont
        % \centering
        \begin{signstabular}[0.7]{p{5cm} >{\centering\arraybackslash}p{10cm} >{\centering\arraybackslash}p{4cm}}
            Студент & \uline{\hfill Ларин Владимир Николаевич  \hfill} \\
            &  \scriptsize (фамилия, имя, отчество)
        \end{signstabular}

        \vspace{\baselineskip}

        \begin{signstabular}[0.7]{p{5cm} >{\centering\arraybackslash}p{10cm} >{\centering\arraybackslash}p{4cm}}
            Группа & \uline{\hfill ИУ7-44Б  \hfill}
        \end{signstabular}

        \vspace{\baselineskip}

        \begin{signstabular}[0.7]{p{5cm} >{\centering\arraybackslash}p{10cm} >{\centering\arraybackslash}p{4cm}}
            Тип практики & \uline{\hfill стационарная  \hfill}
        \end{signstabular}

        \vspace{\baselineskip}

        \begin{signstabular}[0.7]{p{5cm} >{\centering\arraybackslash}p{10cm} >{\centering\arraybackslash}p{4cm}}
            Название предприятия  & \uline{\hfill МГТУ им. Н. Э. Баумана, каф. ИУ7  \hfill}
        \end{signstabular}

        \vspace{\baselineskip}


    \end{table}
    \vfill

    \begin{table}[h!]
        \fontsize{14pt}{0.7\baselineskip}\selectfont
        % \centering
        \begin{signstabular}[0.7]{p{7cm} >{\centering\arraybackslash}p{4cm} >{\centering\arraybackslash}p{4cm}}
            Студент & \uline{\mbox{\hspace*{4cm}}} & \uline{\hfill  Ларин В.Н. \hfill} \\
            & \scriptsize (подпись, дата) & \scriptsize (фамилия, и.о.)
        \end{signstabular}

        \vspace{\baselineskip}

        \begin{signstabular}[0.7]{p{7cm} >{\centering\arraybackslash}p{4cm} >{\centering\arraybackslash}p{4cm}}
            Руководитель практики & \uline{\mbox{\hspace*{4cm}}} & \uline{\hfill Куров А.В. \hfill} \\
            & \scriptsize (подпись, дата) & \scriptsize (фамилия, и.о.)
        \end{signstabular}
        \vspace{\baselineskip}
        \vspace{\baselineskip}

    \end{table}
    \begin{table}[h!]
        \fontsize{14pt}{0.7\baselineskip}\selectfont

        \begin{signstabular}[0.7]{p{2cm} >{\centering\arraybackslash}p{4cm} >{\centering\arraybackslash}p{4cm}}
            Оценка & \uline{\hfill} 
        \end{signstabular}

    \end{table}

    \vfill

    \begin{center}
        \normalsize \textit{\textbf{2021} г.}
    \end{center}
\end{titlepage}



% \begin{executors}
% \personalSignature{Первый исполнитель}{ФИО}

% \personalSignature{Второй исполнитель}{ФИО}
% \end{executors}


%\listoffigures                         % Список рисунков

%\listoftables                          % Список таблиц

%\NormRefs % Нормативные ссылки 
% Команды \breakingbeforechapters и \nonbreakingbeforechapters
% управляют разрывом страницы перед главами.
% По-умолчанию страница разрывается.

% \nobreakingbeforechapters
% \breakingbeforechapters


\thispagestyle{empty}
\begin{center}
    \fontsize{12pt}{0.3\baselineskip}\selectfont \textbf{Министерство науки и высшего образования Российской Федерации \\ Федеральное государственное бюджетное образовательное учреждение \\ высшего образования \\ <<Московский государственный технический университет имени Н.Э. Баумана \\ (национальный исследовательский университет)>> \\ (МГТУ им. Н.Э. Баумана)}

    \fontsize{11pt}{0.3\baselineskip}\selectfont
    \noindent \makebox[\linewidth]{\rule{\textwidth}{4pt}} \makebox[\linewidth]{\rule{\textwidth}{1pt}}
\end{center}
\begin{flushright}
    \fontsize{11pt}{0.5\baselineskip}\selectfont УТВЕРЖДАЮ \\ Заведующий кафедрой \textbf{ИУ7}, \\ \textbf{\hspace*{2.5cm}} \uline{\hspace*{2cm}} \textbf{И.В. Рудаков} \\ <<\uline{\hspace*{1cm}}>> \uline{\hspace*{2.5cm}} \textbf{2021} г.
\end{flushright}

\vfill

\begin{center}
    \fontsize{18pt}{0.7\baselineskip}\selectfont \textbf{З А Д А Н И Е}\\
    \fontsize{16pt}{\baselineskip}\selectfont \textbf{на выполнение курсовой работы}
\end{center}

\normalsize
\begin{flushleft}
    \fontsize{11pt}{0.5\baselineskip}\selectfont
    \setlength{\parskip}{0.1em}
    по дисциплине \uline{\hfill Компьютерная графика \hfill}

    Студент группы \uline{\hfill Ларин Владимир Николаевич \hfill}

    Тема курсовой работы \uline{\hfill ТЕМА \hfill}

    Направленность КР (учебная, исследовательская, практическая, производственная, др.) \\
    \uline{\hfill ТЕМА \hfill}

    Источник тематики (кафедра, предприятие, НИР) \uline{\hfill кафедра \hfill}

    График выполнения работы:  25\% к \uline{?} нед., 50\% к \uline{?} нед., 75\% к \uline{?} нед., 100\% к \uline{?} нед.

    \textbf{\textit{Задание}}
    \uline{\hfill ? \hfill}

    \uline{\hfill ? \hfill}

    \uline{\hfill ? \hfill}

    \uline{\hfill ? \hfill}

    \textbf{\textit{Оформление курсовой работы:}}

    Расчетно-пояснительная записка на 25-30  листах формата А4.
    
    \uline{
        Расчетно-пояснительная записка должна содержать постановку введение, аналитическую часть, конструкторскую часть, технологическую часть, экспериментально-исследовательский раздел, заключение, список литературы, приложения.
    }


    Перечень графического материала (плакаты, схемы, чертежи и т.п.)

    \uline{На защиту проекта должна быть представлена презентация, состоящая из 15-20 слайдов. На слайдах должны быть отражены: постановка задачи, использованные методы и алгоритмы, расчетные соотношения, структура комплекса программ, диаграмма классов, интерфейс, характеристики разработанного ПО, результаты проведенных исследований.
    }

    Дата выдачи задания <<\uline{\hspace*{5mm}}>> \uline{\hspace*{2.5cm}} 20\uline{\hspace*{5mm}} г.
\end{flushleft}


\vfill

\begin{table}[h!]
    \fontsize{11pt}{0.7\baselineskip}\selectfont
    \centering
    \begin{signstabular}[0.7]{p{8cm} >{\centering\arraybackslash}p{3.8cm} >{\centering\arraybackslash}p{3.8cm}}
        \textbf{Руководитель курсовой работы} & \uline{\hspace*{3.8cm}} & \uline{\hfill А.А. Волкова \hfill} \\
        \rule{0pt}{0pt} & \fontsize{9pt}{\baselineskip}\selectfont (Подпись, дата) & \fontsize{9pt}{\baselineskip}\selectfont (И.О. Фамилия)
    \end{signstabular}
    \begin{signstabular}[0.7]{p{8cm} >{\centering\arraybackslash} >{\centering \arraybackslash}p{3.8cm} >{\centering\arraybackslash}p{3.8cm}}
        \textbf{Студент} & \uline{\hspace*{3.8cm}} & \uline{\hfill В.Н. Ларин \hfill} \\
        \rule{0pt}{0pt} & \fontsize{9pt}{\baselineskip}\selectfont (Подпись, дата) & \fontsize{9pt}{\baselineskip}\selectfont (И.О. Фамилия)

    \end{signstabular}
\end{table}
\clearpage

\begin{flushleft}
    \fontsize{11pt}{0.5\baselineskip}\selectfont
    \uline{Примечание:} Задание оформляется в двух экземплярах -- один выдается студенту, второй хранится на кафедре
\end{flushleft}

 
% \include{00-abstract}

\tableofcontents

\printnomenclature % Автоматический список сокращений

\Introduction
Синтез реалистичных изображений в настоящее время переживает смену парадигмы в сторону производства контента в реальном времени. Современные игровые движки способны в режиме реального времени воспроизводить многие реалистичные графические приемы, которые традиционно требовали большого количества времени на вычисления \cite{RONNOW202136}.  
\par
Игровым сценам присущи движения тех или иных объектов, например, бег игрока, полет самолетов и другие. В отличие от снимков движущихся объектов, сделанных фотокамерой, отрисованное изображение будет четким, что лишает зрителя информации о движении объекта. Из-за этого отрисованная анимация объектов может казаться разорванной \cite{Navarro11}.   Данная проблема актуальна, так как рынок видеоигр стремительно развивается, а вычислительной мощности устройств не всегда хватает, чтобы достичь достаточной частоты обновления экрана.
\par
Данную проблему можно решить с помощью добавления на сгенерированное изображение размытия движения. Размытие движения - это эффект, проявляющийся в виде видимых полос, возникающих при движении объекта перед записывающим устройством. Размытие может быть вызвано движением объекта и/или камеры.  Размытие может наблюдаться как на неподвижных фотографиях, так и в последовательностях изображений \cite{Navarro11}.  

Целью работы является проектирование ПО для генерации эффекта размытия движения полигональных моделей. Для достижения поставленной цели необходимо решить следующие задачи:

\begin{itemize}
    \item определить требования для генерации смаза движения;
    \item проанализировать существующие методы генерации смаза движения;
    \item проанализировать существующие методы генерации объемного изображения;
    \item спроектировать генерацию объемного изображения с эффектом размытия движения полигональных моделей;
\end{itemize}



% Проверяем как у нас работают сокращения, обозначения и определения "---
% MAX,
% \Abbrev{MAX}{Maximus ""--- максимальное значение параметра}
% API
% \Abbrev{API}{application programming interface ""--- внешний интерфейс взаимодействия с приложением}
% с обратным прокси.
% \Define{Обратный прокси}{тип прокси-сервера, который ретранслирует}





\mainmatter % это включает нумерацию глав и секций в документе ниже

\chapter{Аналитический раздел}
\label{cha:analysis}
%
% % В начале раздела  можно напомнить его цель
%
% В данном разделе анализируется и классифицируется существующая всячина и пути создания новой всячины. А вот отступ справа в 1 см. "--- это хоть и по ГОСТ, но ведь диагноз же...

\section{Природа размытия движения}

\subsection{Причины появления смаза}

Причины возникновения размытия изображений при съёмке камерой связаны со способом захвата изображения. Свет попадает на светочувствительную матрицу устройства. Затвор и диафрагма влияют на количество света, попавшего на матрицу. Затвор ограничивает время, в течение которого свет попадает на матрицу и формируется итоговое изображение. Данное время называется выдержкой.
\par
После открытия затвора матрица начинает аккумулировать весь свет, попавший на неё. Так как во время движения объекта отраженные лучи света меняют своё положения, то во время выдержки все проекции объекта на плоскость матрицы будут запечатлены на итоговом изображении.    
\par
Данный процесс выдержки может быть формализован с помощью уравнения \eqref{F:F2_1_1} . Во время съемки сцены,  представляет собой содержимое плоскости изображения. Захваченный свет является результатом аккумулирования входящего излучения $L$ в течение времени выдержки $\Delta T$. Функция $f$ моделирует влияние оптики, затвора, диафрагмы и пленки.
\par
\begin{equation}
    I = \int_{\Delta T} f(t) \cdot L(t) dt
    \label{F:F2_1_1}
\end{equation}
\par
Уравнение \eqref{F:F2_1_1} дает представления о характеристиках получаемых изображений. \cite{Navarro11}

\subsection{Описание характеристик смаза}

Выдвинем следующие требования для генерации смазывания движущихся объектов:

\begin{enumerate}
    \item Смазывание должно быть вдоль вектора движения объекта
    \item Продолжительность смаза должна зависеть от скорости движения объекта
    \item Объекты без движения не подвергаются смазыванию
    \item Смаз не терпит разрывов
    \item Смаз не влияет на интенсивность изображения
\end{enumerate}

\section{Методы размытия движения}

\subsection{Размытие движения с помощью накопительного буфера}

Одним из первых решений, которое можно предложить - это склеивание нескольких отрендеренных изображений.
Заметим, что интенсивность изображения должна оставаться без изменений. Тогда получим следующую формулу.


\begin{eqndesc}
    \begin{equation}\label{F:F202}
        I(t, x, y) = \frac{ \sum_{i=0}^{N} {f(\frac{i}{N}) \cdot V_{t- \frac{i}{N} \cdot \Delta T}[x,y] }}
        {\sum_{i=0}^{N} {f(\frac{i}{N}) }}
    \end{equation}
    \\
    где $t - \frac{i}{N} \cdot \Delta T$ "--- момент времени в интервале $[t-\Delta T; t]$,\\
    $f$ "--- функция задающая интенсивность каждого кадра, в определенный момент времени,\\
    $V_t$ "--- Буфер со сгенерированным изображением в момент времени $t$\\
    $I(t,x,y)$ "--- Пиксель $[x,y]$ финального изображения в момент времени $t$
\end{eqndesc}

Для достижения наивысшей скорости отрисовки изображения при генерации видео ряда необходимо хранить результаты последних $N$ кадров. Так как шаг $\Delta T$ и частота обновления экрана $fps$ постоянны, то можно переиспользовать кадры предыдущих итераций. Тогда на каждом новом этапе необходимо отрисовать следущее количество новых кадров  $k = \frac{N}{fps \cdot \Delta T}$, $k \in Z$
\\
Достигается максимальная скорость работы алгоритма при $k = 1$, т.е. когда $fps = \frac{N}{\Delta T}$
Такое решение не строит дополнительные кадры, а берет только $N$ последних отрендеренных кадров, хорошего качества можно добиться только при $N > 32, \Delta T \in [0; 1]$.

\begin{table}[ht]
    \caption{Зависимость частоты кадров от качества и продолжительности размытия движения}
    \begin{tabular}{|r|c|c|c|l|}
        \hline
        $\Delta T$, c  & $N =8$ & $=16$ & $=32$ & $=64$ \\
        \hline
        $1$            & 8      & 16    & 32    & 64    \\
        $\frac{1}{2}$  & 16     & 32    & 64    & 128   \\
        $\frac{1}{4}$  & 32     & 64    & 128   & 256   \\
        $\frac{1}{8}$  & 64     & 128   & 256   & 512   \\
        $\frac{1}{16}$ & 128    & 256   & 512   & 1024  \\
        \hline
    \end{tabular}
    \label{tab:tabular}
\end{table}


Чтобы получить изображение с низкой выдержкой при большом кол-ве семплов нужно иметь большой fps, что становится преградой для реализации размытия в режиме реального времени.

\subsection{Размытие движения с помощью скорости пикселя}

Здесь по статье GPU Gems 3 Nvidea

\subsection{Размытие движения с помощью максимальной скорости движения участка изображения и буфера глубины}

\subsection{Сравнение методов размытия движения}




\section{Удаление невидимых линий и поверхностей}

\subsection{Алгоритм плавающего горизонта}

\subsection{Алгоритм Робертса}

\subsection{Алгоритм Варнока}

\subsection{Алгоритм, использующий список приоритетов}

\subsection{Алгоритм, использующий z-буфер}

\subsection{Алгоритм определения видимых поверхностей путем трассировки лучей}

\subsection{Сравнение алгоритмов удаления невидимых линий и поверхностей}




% Обратите внимание, что включается не ../dia/..., а inc/dia/...
% В Makefile есть соответствующее правило для inc/dia/*.pdf, которое
% берет исходные файлы из ../dia в этом случае.

% \begin{figure}
%   \centering
%   \includegraphics[width=\textwidth]{inc/dia/rpz-idef0}
%   \caption{Рисунок}
%   \label{fig:fig01}
% \end{figure}

% \begin{figure}
%   \centering
%   \includegraphics[height=0.85\textheight]{inc/img/leonardo}
%   \caption{Предполагаемый автопортрет Леонардо да Винчи}
%   \label{fig:leonardo}
% \end{figure}

% В \cite{Pup09} указано, что...

% Кстати, про картинки. Во-первых, для фигур следует использовать \texttt{[ht]}. Если и после этого картинки вставляются <<не по ГОСТ>>, т.е. слишком далеко от места ссылки, "--- значит у вас в РПЗ \textbf{слишком мало текста}! Хотя и ужасный параметр \texttt{!ht} у окружения \texttt{figure} тоже никто не отменял, только при его использовании документ получается страшный, как в ворде, поэтому просьба так не делать по возможности.

% \section{Существующие подходы к созданию всячины}

% Известны следующие подходы...

% \begin{enumerate}
% \item Перечисление с номерами.
% \item Номера первого уровня. Да, ГОСТ требует именно так "--- сначала буквы, на втором уровне "--- цифры.
% Чуть ниже будет вариант <<нормальной>> нумерации и советы по её изменению.
% Да, мне так нравится: на первом уровне выравнивание элементов как у обычных абзацев. Проверим теперь вложенные списки.
% \begin{enumerate}
% \item Номера второго уровня.
% \item Номера второго уровня. Проверяем на длииииной-предлиииииииииинной строке, что получается.... Сойдёт.
% \end{enumerate}
% \item По мнению Лукьяненко, человеческий мозг старается подвести любую проблему к выбору
%   из трех вариантов.
% \item Четвёртый (и последний) элемент списка.
% \end{enumerate}

% Теперь мы покажем, как изменить нумерацию на «нормальную», если вам этого захочется. Пара команд в начале документа поможет нам.

% \renewcommand{\labelenumi}{\arabic{enumi})}
% \renewcommand{\labelenumii}{\asbuk{enumii})}

% \begin{enumerate}
% \item Изменим нумерацию на более привычную...
% \item ... нарушим этим гост.
% \begin{enumerate}
% \item Но, пожалуй, так лучше.
% \end{enumerate}
% \end{enumerate}

% В заключение покажем произвольные маркеры в списках. Для них нужен пакет \textbf{enumerate}.
% \begin{enumerate}[1.]
% \item Маркер с арабской цифрой и с точкой.
% \item Маркер с арабской цифрой и с точкой.
% \begin{enumerate}[I.]
% \item Римская цифра с точкой.
% \item Римская цифра с точкой.
% \end{enumerate}
% \end{enumerate}

% В отчётах могут быть и таблицы "--- см. табл.~\ref{tab:tabular} и~\ref{tab:longtable}.
% Небольшая таблица делается при помощи \Code{tabular} внутри \Code{table} (последний
% полностью аналогичен \Code{figure}, но добавляет другую подпись).

% \begin{table}[ht]
%   \caption{Пример короткой таблицы с коротким названием}
%   \begin{tabular}{|r|c|c|c|l|}
%   \hline
%   Тело      & $F$ & $V$  & $E$ & $F+V-E-2$ \\
%   \hline
%   Тетраэдр  & 4   & 4    & 6   & 0         \\
%   Куб       & 6   & 8    & 12  & 0         \\
%   Октаэдр   & 8   & 6    & 12  & 0         \\
%   Додекаэдр & 20  & 12   & 30  & 0         \\
%   Икосаэдр  & 12  & 20   & 30  & 0         \\
%   \hline
%   Эйлер     & 666 & 9000 & 42  & $+\infty$ \\
%   \hline
%   \end{tabular}
%   \label{tab:tabular}
% \end{table}

% Для больших таблиц следует использовать пакет \Code{longtable}, позволяющий создавать
% таблицы на несколько страниц по ГОСТ.

% Для того, чтобы длинный текст разбивался на много строк в пределах одной ячейки, надо в
% качестве ее формата задавать \texttt{p} и указывать явно ширину: в мм/дюймах
% (\texttt{110mm}), относительно ширины страницы (\texttt{0.22\textbackslash textwidth})
% и~т.п.

% Можно также использовать уменьшенный шрифт "--- но, пожалуйста, тогда уж во \textbf{всей}
% таблице сразу.

% \begin{center}
%   \begin{longtable}{|p{0.40\textwidth}|c|p{0.30\textwidth}|}
%     \caption{Пример длинной таблицы с длинным названием на много длинных-длинных строк}
%     \label{tab:longtable}
%     \\ \hline
%     Вид шума & Громкость, дБ & Комментарий \\
%     \hline \endfirsthead
%     \subcaption{Продолжение таблицы~\ref{tab:longtable}}
%     \\ \hline \endhead
%     \hline \subcaption{Продолжение на след. стр.}
%     \endfoot
%     \hline \endlastfoot
%     Порог слышимости             & 0     &                                                \\
%     \hline
%     Шепот в тихой библиотеке     & 30    &                                                \\
%     Обычный разговор             & 60-70 &                                                \\
%     Звонок телефона              & 80    & \small{Конечно, это было до эпохи мобильников} \\
%     Уличный шум                  & 85    & \small{(внутри машины)}                        \\
%     Гудок поезда                 & 90    &                                                \\
%     Шум электрички               & 95    &                                                \\
%     \hline
%     Порог здоровой нормы         & 90-95 & \small{Длительное пребывание на более
%     громком шуме может привести к ухудшению слуха}                                        \\
%     \hline
%     Мотоцикл                     & 100   &                                                \\
%     Power Mower                  & 107   & \small{(модель бензокосилки)}                  \\
%     Бензопила                    & 110   & \small{(Doom в целом вреден для здоровья)}     \\
%     Рок-концерт                  & 115   &                                                \\
%     \hline
%     Порог боли                   & 125   & \small{feel the pain}                          \\
%     \hline
%     Клепальный молоток           & 125   & \small{(автор сам не знает, что это)}          \\
%     \hline
%     Порог опасности              & 140   & \small{Даже кратковременное пребывание на
%     шуме большего уровня может привести к необратимым последствиям}                       \\
%     \hline
%     Реактивный двигатель         & 140   &                                                \\
%                                  & 180   & \small{Необратимое полное повреждение
%                                  слуховых органов}                                        \\
%     Самый громкий возможный звук & 194   & \small{Интересно, почему?..}                   \\
%   \end{longtable}
% \end{center}

%%% Local Variables:
%%% mode: latex
%%% TeX-master: "rpz"
%%% End:

\chapter{Конструкторский раздел}
\label{cha:design}

В данном разделе рассматриваются реализуемые алгоритмы и методы, приводятся выкладки по модификации существующих.

\section{Генерация мгновенного снимка и подготовка буферов для работы методов размытия движения}

\subsection{Нахождение точек полигона}

Любой выпуклый полигон можно представить в виде пересекающихся прямых. Точки пересечения данных прямых будут являются вершинами выпуклого многоугольника. Каждый пиксель данного многоугольника можно найти, при его сканировании сверху вниз. Необходимо найти абсциссы пересечений сканирующей строки и граней. Чтобы определить с какими гранями нужно искать пересечение, достаточно сравнить ординату сканирующей строки и ординаты точек отрезков: $P_{1y} \leq y \leq P_{2y}$, причем $P_{1y} \leq P_{2y}$. Ребра, удовлетворяющие данному условию, будем называть активными. Для горизонтальных ребер пересечения со сканирующей строкой искать не нужно.

Необходимо заметить, что для выпуклого многоугольника сканирующая строка всегда пересекает ровно два ребра, т.е. существует только два активных ребра. Следовательно, можно заранее сформировать массив ребер и отсортировать его по координатам $y,x$ вершины ребра. 

Для упрощения нахождения пересечений отрезков со сканирующей строкой можно искать значения прямых пошагово. Если задать уравнение прямой формулой $x(y) = ky + m$, то $\frac{\delta x}{\delta y} = k = \frac{P_{1x} - P_{2x}}{P_{1y} - P_{2y}}$. Следовательно $x(y + 1) = x(y) + k$. Т.к. горизонтальные ребра были исключены из рассмотрения, значит $k \neq \pm \infty$. Представим операцию подготовки массива ребер с помощью псевдокода \ref{alg:A_MakeEdges}.

\begin{breakablealgorithm}
    \caption{Подготовка массива ребер}\label{alg:A_MakeEdges}
    
    \begin{algorithmic}[1]
    \Function{MakeEdges}{points}
        \Statex $\triangleright$ $points$ "--- вершины полигона 
        \Statex
        \State $edges \leftarrow$ пустой массив ребер      
        \State $count \leftarrow $ длина массива $points$
        \State $temp \leftarrow$ пустое ребро
        \ForAll{$i \in [0; count]$}
            \State $temp.begin \leftarrow points[i]$
            \State $temp.end \leftarrow points[(i+1)\mod count]$
            \If{ребро $temp$ не горизонтальное}
                \If{$temp.begin.y > temp.end.y$}
                    \State $temp \leftarrow (temp[1], temp[0])$
                \EndIf
    
                \State $temp.k \leftarrow \frac{temp.begin.x - temp.end.x}{temp.begin.y - temp.end.y}$
                \State Добавить $temp$ в массив $edges$
            \EndIf
        \EndFor
        
        \State отсортировать массив $edges$ по координатам $y$ и $x$ первой вершины ребра
    
        \State \Return $edges$
    
    \EndFunction
    \end{algorithmic}
\end{breakablealgorithm}

Далее из данного массива необходимо получать активные ребра по мере просмотра многоугольник сверху вниз. 



Для правильной работы алгоритма удаления невидимых линий и поверхностей, также для каждой точки полигона необходимо найти координату $z$. Её можно получить из уравнения плоскости: $Ax + By + Cz + D = 0$, которое находится по трем точкам, не лежащих на одной прямой.

Путь $P_1$, $P_2$, $P_3$ - вершины полигона, не лежащие на одной прямой, тогда найдем два вектора, определяющих плоскость полигона: $\vec{a} = P_1 - P_2 $, $\vec{b} = P_1 - P_3 $. Из свойств векторного произведения: векторное произведение $\vec{a} \times \vec{b}$  есть нормаль к плоскости, образованной векторами $\vec{a}$ и $\vec{b}$. Также известно, что коэффициенты $A$, $B$, $C$  в уравнение плоскости задают нормаль к данной плоскости $ \vec{n} = \begin{pmatrix}
        A & B & C
    \end{pmatrix}$. Векторное произведение можно найти по формуле \eqref{F:vector_scalar}.

\begin{equation}
    \label{F:vector_scalar}
    \vec{n} = \vec{a} \times \vec{b} = \begin{pmatrix}
        \vec{a}_y \vec{b}_z - \vec{a}_z \vec{b}_y &
        \vec{a}_z \vec{b}_x - \vec{a}_x \vec{b}_z &
        \vec{a}_x \vec{b}_y - \vec{a}_y \vec{b}_x
    \end{pmatrix}
\end{equation}

Оставшийся коэффициент $D$ находим, подставляя произвольную точку плоскости в уравнение данной плоскости. С учетом данного факта получим уравнение функции \eqref{F:F_ploskost_koef} нахождения коэффициентов плоскости по точке данной плоскости и двух векторов, лежащих в этой плоскости.


\begin{equation}
    \label{F:F_ploskost_koef}
    FindSurf(P, \vec{a}, \vec{b}) =
    \begin{pmatrix}
        A \\
        B \\
        C \\
        D
    \end{pmatrix} =
    \begin{pmatrix}
        \vec{a}_y \vec{b}_z - \vec{a}_z \vec{b}_y \\
        \vec{a}_z \vec{b}_x - \vec{a}_x \vec{b}_z \\
        \vec{a}_x \vec{b}_y - \vec{a}_y \vec{b}_x \\
        -A \cdot P_x - B \cdot P_y - C \cdot P_z
    \end{pmatrix}
\end{equation}

Зная точки $x$, $y$ полигона и коэффициенты плоскости можно найти координату $z$. Обозначим данную функцию как $FindZ(surf, point)$, представленную на формуле \eqref{F:F_FindZ}.

\begin{equation}
    \label{F:F_FindZ}
    FindZ(surf, point) = -\frac{surf_A \cdot point_x + surf_B \cdot point_y + surf_D}{surf_C}
\end{equation}

Для упрощения процесса нахождения координаты $z$ во время прохода по сканирующей строке можно считать значение пошагово по формуле \eqref{F:F_z_iterative_find_depth}.

\begin{eqndesc}
    \begin{equation}
        \label{F:F_z_iterative_find_depth}
        FindNextZ(k, z) = FindZ(surf, point)  + \frac{\delta (FindZ(surf, point))}{\delta x} =
        z + k
    \end{equation}
    , где
    $k = \frac{A}{C}$ "--- приращение $z$ для плоскости $surf$
\end{eqndesc}

\subsection{Нахождение вектора скорости пикселя}

Для работы методов размытия необходим буфер скоростей пикселей. Заметим, что скорость $v = \begin{pmatrix}
        {\Delta x} & {\Delta y}
    \end{pmatrix}$ пикселя задается в плоскости изображения, т.е. с учетом преобразований камеры.

Так как любое перемещение объекта задается изменением положения, поворота и масштаба тела относительно его центра, то данное перемещение можно представить с помощью последовательного произведения матриц преобразования, указанных в таблице \ref{tab:tranformations_table}. Данную матрицу будем задавать как $Transf(c, \vec{t}, \vec{k}, \vec{r})$, где $c$ "--- центр тела, $\vec{t}$ "--- перемещение тела, $\vec{k}$ "--- увеличение тела, $r$ "--- поворот тела. Вектора указаны в системе координат сцены.
\begin{center}
    \begin{longtable}{|p{0.115\textwidth}|p{0.41\textwidth}|p{0.43\textwidth}|}
        \caption{Необходимые преобразования и порядок их применения}
        \label{tab:tranformations_table}
        \\ \hline
        № применения & Матрица преобразования                    & Описание преобразования                                                    \\
        \hline \endfirsthead
        \subcaption{Продолжение таблицы~\ref{tab:tranformations_table}}
        \\ \hline \endhead
        \hline \subcaption{Продолжение на след. стр.}
        \endfoot
        \hline \endlastfoot
        1            & $
            \begin{pmatrix}
                1   & 0   & 0   & 0 \\
                0   & 1   & 0   & 0 \\
                0   & 0   & 1   & 0 \\
                c_x & c_y & c_z & 1 \\
            \end{pmatrix}
        $            & Перенос начала координат в центр тела $c$                                                                              \\
        \hline
        2            & $\begin{pmatrix}
                1 & 0              & 0               & 0 \\
                0 & \cos(\alpha_x) & -\sin(\alpha_x) & 0 \\
                0 & \sin(\alpha_x) & \cos(\alpha_x)  & 0 \\
                0 & 0              & 0               & 1 \\
            \end{pmatrix}$               & Поворот тела вдоль оси $x$ на угол $\alpha_x$                              \\
        \hline
        3            & $\begin{pmatrix}
                \cos(\alpha_y) & 0 & -\sin(\alpha_y) & 0 \\
                0              & 1 & 0               & 0 \\
                \sin(\alpha_y) & 0 & \cos(\alpha_y)  & 0 \\
                0              & 0 & 0               & 1 \\
            \end{pmatrix}$              & Поворот тела вдоль оси $y$ на угол $\alpha_y$                              \\
        \hline
        4            & $\begin{pmatrix}
                \cos(\alpha_z) & -\sin(\alpha_z) & 0 & 0 \\
                \sin(\alpha_z) & \cos(\alpha_z)  & 0 & 0 \\
                0              & 0               & 0 & 1 \\
                0              & 0               & 1 & 0
            \end{pmatrix}$              & Поворот тела вдоль оси $z$ на угол $\alpha_z$                              \\
        \hline
        5            & $\begin{pmatrix}
                k_x & 0   & 0   & 0 \\
                0   & k_y & 0   & 0 \\
                0   & 0   & k_z & 0 \\
                0   & 0   & 0   & 1 \\
            \end{pmatrix}$              & Выполнение масштабирования тела на $k_x$, $k_y$, $k_z$                     \\
        \hline
        6            & $\begin{pmatrix}
                1        & 0        & 0        & 0 \\
                0        & 1        & 0        & 0 \\
                0        & 0        & 1        & 0 \\
                dx - c_x & dy - c_y & dz - c_z & 1
            \end{pmatrix}$              & Выполнение перемещения тела на $dx$, $dy$, $dz$ и возврат начала координат \\


        \hline
    \end{longtable}
\end{center}
Для более структурированного повествования для каждого тела $object$, введем их параметры в виде функций, зависящих от времени:
\begin{itemize}
    \item $object.pos(t)$ "--- положение центра тела в момент времени $t$;
    \item $object.rot(t)$ "--- поворот тела относительно центра в момент $t$;
    \item $object.scale(t)$ "--- масштаб тела относительно центра в момент $t$.
\end{itemize}

Следовательно, зная масштаб $object.scale$, положение $object.pos$, поворот $object.rot$ некоторого тела $object$ в моменты времени $t_0$ и $t_1$, где время выдержки $\Delta T = t_1 - t_0 > 0$, можно найти положения вокселя $p = (x,y,z)$ в координатах сцены через $\Delta T$ с помощью матрицы преобразования, представленной на формуле \eqref{F:F_ObjectMatrix}.


\begin{equation}
    \label{F:F_ObjectMatrix}
    \begin{matrix}
        {
            ObjectMatrix(object, t, \Delta T) =
            Transf(
            object.pos(t),
        } \\

        {
        object.pos(t + \Delta T) - object.pos(t),

        }
        {\frac{object.scale(t + \Delta T)}{object.scale(t)}, }\\{  object.rot(t + \Delta T) - object.rot(t))}
    \end{matrix}
\end{equation}


Чтобы перейти в координаты сцены из системы координат изображения, необходимо умножить координаты вокселя $p$ на обратную матрицу преобразования камеры $(Cam(cam, t_0))^{-1} = ICam(cam, t_0)$ в момент времени $t_0$. Чтобы вернуться в систему координат изображения, необходимо домножить на матрицу преобразования камеры $Cam(cam, t_0 + \Delta T)$ в момент времени $t_0 + \Delta T$.

Тогда можем получить координаты вокселя $p_1$ в пространстве изображения через время выдержки $\Delta T$. Запишем преобразования \eqref{F:F_voxel_transf} нахождения положения вокселя $p$ через время выдержки $\Delta T$ какого-то тела.

\begin{equation}
    \label{F:F_voxel_transf}
    \begin{matrix}
        {VoxelTransf(Cam, ICam, ObjectMatrix) =} \\
        {= ICam(cam, t_0) \times ObjectMatrix(object, t, \Delta T) \times Cam(cam, t_0 + \Delta T)}
    \end{matrix}
\end{equation}

Зная положения вокселя $p$ в моменты времени начала и окончания выдержки возможно найти скорость пикселя по формуле \eqref{F:F_voxel_speed}.

\begin{eqndesc}
    \begin{equation}
        \label{F:F_voxel_speed}
        FindV(p_0, VoxelTransf) = p_0 - p_0 \times VoxelTransf(Cam, ICam, ObjectMatrix)
    \end{equation}
\end{eqndesc}
\subsection{Учет положения и угла поворота камеры}
В предыдущем пункте было упомянуто о матрице преобразования камеры $Cam(cam, t)$, вывод которой будет рассмотрен в данном пункте.

Камера $cam$ также является объектом, но кроме параметров, приведенных в предыдущем пункте, также содержит коэффициент перспективы $cam.k$.

Можно задать преобразование, которое перемещает систему координат сцены в систему координат камеры, через матрицу $Cam(t)$, заданную формулой \eqref{F:F_Cam_Matrix}.

\begin{eqndesc}
    \begin{equation}
        \label{F:F_Cam_Matrix}
        \begin{matrix}
            {Cam(cam, t) = Transf(cam.pos(t), -cam.pos(t),} \\{ cam.scale(t), cam.rot(t)) \times Persp(k)}
        \end{matrix}
    \end{equation}
    , где
    $Persp(k)$ "--- перспективное преобразование, заданное формулой \eqref{F:F_persp_matrix}\\
\end{eqndesc}

\begin{equation}
    \label{F:F_persp_matrix}
    Persp(k) = \begin{pmatrix}
        1 & 0 & 0 & 0 \\
        0 & 1 & 0 & 0 \\
        0 & 0 & 1 & k \\
        0 & 0 & 0 & 1 \\
    \end{pmatrix}
\end{equation}

Также для вычислений необходима обратная матрица преобразования камеры $(Cam(cam, t))^{-1}$. Её можно задать формулой \eqref{F:F_inv_cam_matrix}

\begin{equation}
    \label{F:F_inv_cam_matrix}
    \begin{matrix}
        {ICam(cam, t) = (Cam(cam, t))^{-1} = Persp(-k) \times} \\
        {\times Transf(\vec{0}, cam.pos(t), \frac{1}{cam.scale(t)}, -cam.rot(t))}
    \end{matrix}
\end{equation}

\subsection{Модификация алгоритма построения изображения с помощью z-буфера}


Данный алгоритм для каждого кадра буфер изображения $V$ заполняет цветом фона, буфер глубины $Z$ - максимальной величиной (бесконечностью), буфер скоростей пикселей $v$ зануляет. 

Алгоритм подразумевает предварительное формирование матрицы преобразования камеры $CamM = Cam(cam, t)$ и обратной матрицы $ICamM = ICam(cam, t + \Delta T)$, чтобы переходить из системы координат камеры в систему координат сцены и обратно. Также для каждой модели $model$ необходимо найти преобразование вокселя $VoxelTransf(Cam, ICam, ObjectMatrix(model, t, \Delta T))$ за время выдержки $\Delta T$.  

Для отрисовки единичного полигона представлен псевдокод \ref{alg:A_DrawPolygon}.  На вход данной функции поступают $p$ "--- вершины полигона, $color$ "--- цвет полигона, $V$, $z$, $v$ "--- буферы кадра, глубины и скорости соответственно и матрица преобразования вокселя $VoxelTransfM$.


\begin{breakablealgorithm}
\caption{Модифицированный алгоритм удаления невидимых линий и поверхностей с помощью z-буфера и формирования буфера скоростей} \label{alg:A_DrawPolygon}

\begin{algorithmic}[1]
\Function{DrawPolygon}{$V, Z, v, p, color, VoxelTransfM$}
    \State 
        $surf \leftarrow FindSurf(
            p[0],
            p[0] - p[1],
            p[0] - p[2]
        )$
        \Comment коэффициенты плоскости полигона для поиска глубины
    \State $kz \leftarrow \frac{surf.A}{surf.C}$    
    \Statex

    \State $edges \leftarrow MakeEdges(p)$ \Comment см. псевдокод \ref{alg:A_MakeEdges} 
    
    \State $edgeL \leftarrow edges[0]$ 
    \State $edgeR \leftarrow edges[1]$
    \State $last \leftarrow 1$ 
    \Statex
    \State $xL \leftarrow edgeL.begin.x$ 
    \State $xR \leftarrow edgeR.begin.x$

    \Statex \Comment т.к. массив ребер отсортирован, то в силу выпуклости полигона получим минимальное и максимальное значения $y$
    \State $y_{min} \leftarrow edges[0].begin.y$
    \State $y_{max} \leftarrow edges[$ последний элемент $].end.y$ 

    \ForAll{$y \in [y_{min}, y_{max}]$} 
        
        \State $z \leftarrow FindZ(x,y, surf)$ \Comment см. формулу \eqref{F:F_FindZ} 
        \ForAll{$x \in [xL, xR]$} 
            \If{$z < Z[x,y]$}
                \State $V[x,y] \leftarrow color$
                \State $Z[x,y] \leftarrow z$
                \State $v[x,y] \leftarrow FindV((x,y,z,1), VoxelTransfM)$ 
                \Statex \Comment см. формулу \eqref{F:F_voxel_speed} 
            \EndIf
            \State $z \leftarrow FindNextZ(kz, z)$ \Comment см. формулу \eqref{F:F_z_iterative_find_depth} 
        \EndFor
        \State $xL \leftarrow xL + kL$
        \State $xR \leftarrow xR + kR$
        \If{$edgeL.end.y \leq y$}
            \State $last \leftarrow last + 1$
            \State $edgeL \leftarrow edges[last]$
            \State $xL \leftarrow edgeL.begin.x$ 
        \EndIf
        
        \If{$edgeR.end.y \leq y$}
            \State $last \leftarrow last + 1$
            \State $edgeR \leftarrow edges[last]$
            \State $xR \leftarrow edgeR.begin.x$ 
        \EndIf

    \EndFor
    
\EndFunction
\Statex 

\end{algorithmic}
\end{breakablealgorithm}

Для получения финального мгновенного кадра с решением проблемы удаления невидимых линий и поверхностей, а также поиска буферов глубины и скоростей необходимо данную операцию применить для каждого полигона каждой модели.

\section{Реализация методов размытия движения изображения}

\subsection{Размытие движения с помощью накопительного буфера}

Данный метод требует создания $N$ мгновенных снимков $shots$ для получения смазанного изображения. Метод достаточно простой в реализации, но требует больших вычислительных затрат по генерации мгновенных снимков. Реализация данного метода для обработки единичного пикселя представлена на псевдокоде \ref{alg:A_AccumulateBufferBlur}.

% Рекомендуется подбирать количество снимков и длину выдержки так, чтобы $N$ был кратен $fps \cdot \Delta T$, где $fps$ - частота выходного видео потока, чтобы можно было переиспользовать предыдущие снимки.

\begin{breakablealgorithm}
    \caption{Размытие движения с помощью накопительного буфера} \label{alg:A_AccumulateBufferBlur}
    \begin{algorithmic}[1]
        \Function{AccumulatePixelBlur}{$pixel$, $shots$}
        \State $color \leftarrow 0$
        \ForAll{$i \in [0; N)$}
            \State $color \leftarrow color + shots[i][pixel]$
        \EndFor
        \State \Return $color / N$
        \EndFunction
\end{algorithmic}
\end{breakablealgorithm}

\subsection{Размытие движения с помощью скорости пикселя}

Данный метод требует для своей работы наличия буфера скоростей $v$, одного мгновенного снимка (буфера кадра $V$) и количество анализируемых пикселей $N$.  На выходе метода получим буфер кадра со смазом $V_{out}$. Реализация данного метода для единичного пикселя представлена на псевдокоде \ref{alg:A_PerPixelBlur}.

\begin{breakablealgorithm}
    \caption{Размытие движения с помощью скорости пикселя} \label{alg:A_PerPixelBlur}
    \begin{algorithmic}[1]
        \Function{PixelVelocityBlur}{$pixel, V, v$}
            \State $temp \leftarrow pixel$
            \State $velocity \leftarrow v[temp] / N$
            \State $color \leftarrow V[temp]$
            \ForAll{$i \in [0; N)$}
                \State $temp \leftarrow temp + velocity$
                \State $color \leftarrow color + V[temp]$
            \EndFor
            \State \Return $color / N$
        \EndFunction
\end{algorithmic}
\end{breakablealgorithm}




\subsection{Размытие движения с помощью максимальной скорости движения участка изображения и буфера глубины}

Данный алгоритм является модификацией предыдущего метода, но в отличии от предыдущего требует для своей работы предварительное формирования буфера максимальной скорости движения участка изображения $v_{neighbor}$ по формуле \eqref{F:F2_3_2} и буфера глубины $Z$. Реализация данного метода для единичного пикселя представлена на псевдокоде \ref{alg:A_McGuireBlur}.

\begin{breakablealgorithm}
\caption{Размытие движения с помощью максимальной скорости движения участка изображения и буфера глубины} \label{alg:A_McGuireBlur}
\begin{algorithmic}[1]
    \Function{McGuireBlur}{$pixel$, $V$, $Z$, $v$, $v_{neighbor}$, }
        \State $A \leftarrow (x,y)$
        \State $velocity \leftarrow v_{neighbor}[A/k]$
        \If{$velocity < \varepsilon$} \Comment Нет размытия 
            \State \Return $V[A]$
        \EndIf
        
        \State $velocity \leftarrow velocity / N$
        \State $weight \leftarrow \frac{1}{|velocity[A]|}$
        \State $color \leftarrow V[A] \cdot weight$
        
        \State $B \leftarrow A$
        \ForAll{$i \in [0; N)$}
            \State $B \leftarrow B + velocity$
            \Statex \Comment Тест принадлежности переднему и заднему фону с помощью z-буфера
            \State $fg \leftarrow softDepthCompare(Z[A],Z[B])$  \Comment см. пункт \ref{cha:analysis_mcgiure} 
            \State $bg \leftarrow softDepthCompare(Z[B],Z[A])$
            \Statex
            \State $w \leftarrow  fg \cdot cone(B,A,v[B])$
            \State $w \leftarrow  w + bg \cdot cone(A,B,v[A])$
            \State $w \leftarrow  w + cylinder(A,B,v[A]) \cdot  cylinder(B,A,v[B]) \cdot 2$
            \Statex
            \State $color \leftarrow color + w \cdot V[Y]$
            \State $weight \leftarrow weight + w$
        \EndFor
        \State \Return $color / weight$
    \EndFunction        
\end{algorithmic}
\end{breakablealgorithm}



\section{Архитектура приложения}

В данном пункте будут рассмотрены детали архитектуры приложения и возможности её будущей модификации для расширения функционала.

\subsection{Сцена}

Приложение имеет возможность добавить на сцену несколько полигональных моделей и задать их перемещения. Диаграмма классов сцены представлена на рисунке \ref{fig:uml_scene}.

\begin{figure}[h]
    \centering
    \includesvg[width=1\columnwidth,inkscapelatex=false]{img/umls/svg/Model!SceneObject_1.svg}
    \caption{Диаграмма классов сцены}
    \label{fig:uml_scene}
\end{figure}

Класс сцены Scene является контейнером, который хранит объекты сцены. Объект сцены представлен интерфейсом ISceneObject, который предоставляет методы для задания и получения позиции, поворота, размера объекта в любой момент времени, цвета объекта и получения типа объекта. Для расширения функционала объектов сцены была предусмотрена реализация паттерна Посетитель (IVisitor).

Класс AnimatedCoordProperty предоставляет возможность задавать начальные и конечные параметры объекта, задаваемые с помощью трехмерного вектора и изменяемые линейно с течением времени. С помощью композиции экземпляров данного класса задаются функции позиции, поворота и масштаба тела от времени.

На диаграмме представлено два типа объектов: Модель и Камера. Класс камеры Camera предоставляет два метода для получения прямого и обратного преобразований камеры в определенный момент времени. Класс модели Model является контейнером полигонов (Polygon), который является в свою очередь контейнером точек (IPoint).   

Данная структура классов позволяет добавлять новые типы объектов и новый функционал к ним. 

\subsection{Отрисовка сцены и размытие движения}
Было предложено решать проблему формирования изображения сцены в определенный момент с помощью следующих классов. На рисунке \ref{fig:uml_render} представлена диаграмма классов отрисовки сцены.  

\begin{figure}[h]
    \centering
    \includesvg[width=1\columnwidth,inkscapelatex=false]{img/umls/svg/Model!Renderdiagram_2.svg}
    \caption{Диаграмма классов отрисовки сцены}
    \label{fig:uml_render}
\end{figure}

Для управления процесса отрисовки был спроектирован класс менеджера отрисовки RenderManager, который формирует изображение, буфер глубины и буфер  скоростей пикселей при необходимости. Для отрисовки используется интерфейс отрисовщика IDrawer, который является адаптером для движков построения изображения (например, SFML, Qt Painter или OpenGL). Для движков был подготовлен абстрактный класс BaseDrawer, который реализует обработку преобразований и их применение при отрисовке полигонов.  

Класс преобразований ITransformation, состоит из методов применения преобразования к точке, объединения нескольких преобразований и получения копии преобразования и абстрактной фабрики преобразований.
Класс абстрактной фабрики преобразований ITransformationFactory создаёт преобразования перемещения, поворота, масштабирования, перспективы и единичное преобразования. С помощью объедения данных преобразований можно получить любое преобразование, необходимое в рамках работы программы.

Класс контекста отрисовки RenderContext содержит буферы, которые будут генерироваться и изменяться в процессе отрисовки изображения и использоваться во время генерации размытия изображения.

Для обработки структуры объектов сцены и ее отрисовки используется класс посетителя отрисовки RenderVisitor. Он также позволяет сформировать пространственное преобразование объекта в определенный момент времени.

Размытие движения реализуется независимо от построения мгновенного снимка. На рисунке \ref{fig:uml_render} представлена диаграмма классов размытия движения.  

\begin{figure}[h]
    \centering
    \includesvg[width=0.9\columnwidth,inkscapelatex=false]{img/umls/svg/Model!BlurDiagram_3.svg}
    \caption{Диаграмма классов размытия движения}
    \label{fig:uml_render}
\end{figure}

Для выполнения размытия движения используется паттерн стратегия. Он был выбран, так как позволяет быстро добавлять новые стратегии (методы смаза) в приложение. Интерфейс стратегии размытия IBlurStrategy имеет методы выполнения смаза и получения названия стратегии. Каждая стратегия смаза соответствует методам размытия изображения: с помощью накопительного буфера (AccumulateBlurStrategy), с помощью скорости пикселя (PixelBlurStrategy) и с помощью максимальной скорости
движения участка изображения и буфера глубины (NeiborBlurStrategy). Стратегия использует параметры размытия BlurConfig, которые содержат менеджер отрисовки, с помощью которого производится формирование нужных данных для определенного метода смаза.

\section{Вывод}


В данном разделе были рассмотрены этапы модификации алгоритма удаления невидимых линий и поверхностей с помощью буфера глубины и методы построения размытия движения, была спроектирована архитектура приложения. 


%%% Local Variables:
%%% mode: latex
%%% TeX-master: "rpz"
%%% End:

\chapter{Технологический раздел}
\label{cha:impl}
В данном разделе представлены средства разработки программного
обеспечения, детали реализации.

\section{Средства реализации}

Основным средством разработки является язык программирования. Был выбран язык программирования C++. Данный выбор обоснован высокой скоростью работы языка, поддержкой объектно-ориентированного подхода программирования и строгой типизаций \cite{cpplang}. Для написания шейдеров использован язык программирования GLSL, который поддерживается большинством современных видеокарт.

В связи с гибкой архитектурой приложения возможна быстрая интеграция любой библиотеки графического интерфейса (GUI). Было принято решение использовать Free GLUT и ImGui, которые предоставляет базовый набор графических компонентов и функций создания окна,  поддерживают платформы Linux, Windows, macOS и другие \cite{imgui}. 

Для поддержания качества кода было принято использовать инструменты статического анализа кода cpplint\cite{cpplint} и cppcheck\cite{cppcheck}, отладчик использования памяти valgrind\cite{valgrind}. 

\section{Детали реализации алгоритмов}

\subsection{Формирование изображения, буфера глубины и скорости}

Для формирования формирования буферов изображения, глубины и скорости используется шейдер, представленный на листинге \ref{lst:zbuffer.vs}, реализующий модификацию алгоритма удаления невидимых плоскостей с помощью z буфера. Для работа шейдера в него подаются точки полигона, z буфер, буфер скоростей, буфер кадра, ширина и высота буферов, цвет закраски полигона, матрица преобразования в момент времени $t$, обратная матрица преобразований в момент времени $t$, матрица преобразования в момент времени $t - \Delta T$, где $\Delta T$ - время выдержки.
% \begin{figure}[H]    
        \lstinputlisting[
                language=C,
                caption=Шейдер формирования изображения\, буфера глубины и скорости для единичного полигона,
                label=lst:zbuffer.vs,
                ]{code/shaders/zbuffer.cl}
% \end{figure}

% \subsection{Пространственные преобразования}

% В графическом модуле реализованы основные манипуляции со сценой, в том числе и формирование преобразований для каждого объекта и активной камеры. На листинге \ref{lst:cameraTransf} представлено формирование преобразования камеры. Для формирования преобразования анализируется положение и поворот камеры в пространстве сцены в определенный момент времени. 

% \lstinputlisting[language=C,caption=Формирование преобразования камеры,label=lst:cameraTransf]{code/other/cameraTransf.cpp}

% На листинге \ref{lst:cameraTransf} представлено формирование преобразования объекта. Для формирования преобразования анализируется положение, поворот, и масштабирование объекта в пространстве сцены в определенный момент времени. 

% \lstinputlisting[language=C,caption=Формирование преобразования объекта,label=lst:objectTransf]{code/other/objectTransf.cpp}

% \subsection{Буфер скорости пикселей}

% Для формирования буфера скорости пикселей используется конвейер из вершинного и фрагментного шейдеров. Вершинный шейдер вычисляет положение вершины в два момента времени, соответствующие началу и концу выдержки изображения. Данный шейдер представлен на листинге \ref{lst:velocity.vs}.

% \begin{figure}[H]    
%         \lstinputlisting[
%                 language=C,
%                 caption=Вершинный шейдер формирования буфера скорости,
%                 label=lst:velocity.vs,
%                 ]{code/shaders/velocity.vs.glsl}
% \end{figure}


% Полученные положения интерполируются по всему полигону и для каждого пикселя полигона вычисляется разница начального и конечного положения пикселя в пространстве экрана c помощью фрагментного шейдера. Данная разница является скоростью пикселя и заносится в результирующий буфер. Данный шейдер представлен на листинге \ref{lst:velocity.fs}. 



% \lstinputlisting[language=C,caption=Фрагментный шейдер формирования буфера скорости,label=lst:velocity.fs]{code/shaders/velocity.fs.glsl}



\subsection{Формирования смаза с помощью скорости пикселя}

Для формирования смаза с помощью скорости пикселя используются буфер изображения и буфер скорости пикселей, полученные ранее. Для применения фрагментного шейдера для каждого пикселя буфера изображения отрисовывается единичный полигон, занимающий всё пространство экрана. Также в шейдер передается константа $S$ - максимальное количество анализируемых пикселей для формирования смаза. Данный шейдер представлен на листинге \ref{lst:pixelblur.fs}. 

\lstinputlisting[language=C,caption=Фрагментный шейдер формирования смаза с помощью скорости пикселя,label=lst:pixelblur.fs]{code/shaders/pixelBlur.fs.glsl}


\subsection{Формирования смаза помощью максимальной скорости движения участка изображения и буфера глубины}

Формирования буфера максимальной скорости движения участков изображения происходит в два этапа. На первом этапе формируется буфер скорости движения участков изображения. В данный шейдер передается буфер скорости, сформированный ранее и размер участка $k \times k$ в пикселях. Данный шейдер представлен на листинге \ref{lst:tile.fs}. 


\lstinputlisting[language=C,caption=Фрагментный шейдер формирования буфера скорости движения участков изображения,label=lst:tile.fs]{code/shaders/tileMax.fs.glsl}

Следующим этапом формируется  буфер максимальной скорости движения участков изображения с помощью ранее вычисленного буфера  скорости движения участков изображения. Фрагментный шейдер представлен на листинге \ref{lst:neighborMax.fs}.

\lstinputlisting[language=C,caption=Фрагментный шейдер формирования буфера максимальной скорости движения участка изображения,label=lst:neighborMax.fs]{code/shaders/neighborMax.fs.glsl}

Фрагментный шейдер формирования смаза с помощью максимальной скорости движения участка изображения и буфера глубины представлен на листинге \ref{lst:gatherAll.fs}.


\lstinputlisting[language=C,caption=Фрагментный шейдер формирования смаза с помощью максимальной скорости движения участка изображения и буфера глубины,label=lst:gatherAll.fs]{code/shaders/gatherAll.fs.glsl}



\section{Графический интерфейс}

Был разработан графический интерфейс приложения, который представлен на рисунке \ref{fig:gui_window}. 

\begin{figure}[h]
    \centering
    \includegraphics[width=0.9\columnwidth]{img/gui/common_new.png}
    \caption{Общий вид окна}
    \label{fig:gui_window}
\end{figure}


В правой половине окна расположен предпросмотр сцены. В левой половине список объектов сцены и параметры в виде вкладок. На каждой вкладке можно настроить начальное и конечные время анимации и поворот, масштаб, перемещение объекта.
Интерфейс предоставляет возможность просмотреть один из следующих выводов программы:
\begin{itemize}
    \item размытие движения с помощью скорости пикселя;
    \item размытие движения с помощью максимальной скорости
    движения участка изображения и буфера глубины;
    \item буфер кадра -- изображение без размытия;
    \item буфер глубины;
    \item буфер скорости пикселей;
    \item буфер скорости движения участка изображения;
    \item буфер максимальной скорости движения участка изображения.
\end{itemize}

На рисунке \ref{fig:result_buffers} представлены визуализации буферов.



\begin{figure}[h]
    \centering
    \begin{minipage}[h]{0.49\linewidth}
        \center{\includegraphics[width=0.8\linewidth]{img/screenshots/t1/color.png} \\ а)}
    \end{minipage}
    \hfill
    \begin{minipage}[h]{0.49\linewidth}
        \center{\includegraphics[width=0.8\linewidth]{img/screenshots/t1/depth.png} \\ б)}
    \end{minipage}

    \begin{minipage}[h]{0.49\linewidth}
        \center{\includegraphics[width=0.8\linewidth]{img/screenshots/t1/velocity.png} \\ в)}
    \end{minipage}
    \hfill
    \begin{minipage}[h]{0.49\linewidth}
        \center{\includegraphics[width=0.8\linewidth]{img/screenshots/t1/neibor.png} \\ г)}
    \end{minipage}


    \begin{minipage}[h]{0.49\linewidth}
        \center{\includegraphics[width=0.8\linewidth]{img/screenshots/t1/maxneibor.png} \\ д)}
    \end{minipage}
    \hfill
    \begin{minipage}[h]{0.49\linewidth}
        \center{\includegraphics[width=0.8\linewidth]{img/screenshots/t1/pixelblur.png} \\ е)}
    \end{minipage}

    \begin{minipage}[h]{0.49\linewidth}
        \center{\includegraphics[width=0.8\linewidth]{img/screenshots/t1/mcguire.png} \\ ж)}
    \end{minipage}


    \caption{Результат работы программы. \\ а) Буфер кадра б) Буфер глубины в) Буфер скорости г) Буфер скорости участка \\ д) Буфер максимальной скорости участков  e) Размытие по скорости пикселя \\ ж) Размытие по максимальной скорости участка и буферу скорости  }
    \label{fig:result_buffers}
\end{figure} 


% В данном разделе описано изготовление и требование всячины. Кстати,
% в Latex нужно эскейпить подчёркивание (писать <<\verb|some\_function|>> для \Code{some\_function}).

% \ifPDFTeX
% Для вставки кода есть пакет \Code{listings}. К сожалению, пакет \Code{listings} всё ещё
% работает криво при появлении в листинге русских букв и кодировке исходников utf-8.
% В данном примере он (увы) на лету конвертируется в koi-8 в ходе сборки pdf.

% Есть альтернатива \Code{listingsutf8}, однако она работает лишь с
% \Code{\textbackslash{}lstinputlisting}, но не с окружением \Code{\textbackslash{}lstlisting}

% Вот так можно вставлять псевдокод (питоноподобный язык определен в \Code{listings.inc.tex}):

% \begin{lstlisting}[style=pseudocode,caption={Алгоритм оценки дипломных работ}]
% def EvaluateDiplomas():
%     for each student in Masters:
%         student.Mark := 5
%     for each student in Engineers:
%         if Good(student):
%             student.Mark := 5
%         else:
%             student.Mark := 4
% \end{lstlisting}

% Еще в шаблоне определен псевдоязык для BNF:

% \begin{lstlisting}[style=grammar,basicstyle=\small,caption={Грамматика}]
%   ifstmt -> "if" "(" expression ")" stmt |
%             "if" "(" expression ")" stmt1 "else" stmt2
%   number -> digit digit*
% \end{lstlisting}

% В листинге~\ref{lst:sample01} работают русские буквы. Сильная магия. Однако, работает
% только во включаемых файлах, прямо в \TeX{} нельзя.

% % Обратите внимание, что включается не ../src/..., а inc/src/...
% % В Makefile есть соответствующее правило для inc/src/*,
% % которое копирует исходные файлы из ../src и конвертирует из UTF-8 в KOI8-R.
% % Кстати, поэтому использовать можно только русские буквы и ASCII,
% % весь остальной UTF-8 вроде CJK и египетских иероглифов -- нельзя.

% \lstinputlisting[language=C,caption=Пример (\Code{test.c}),label=lst:sample01]{inc/src/test.c}

% \else

% Для вставки кода есть пакет \texttt{minted}. Он хорош всем кроме: необходимости Python (есть во всех нормальных (нет, Windows, я не про тебя) ОС) и Pygments и того, что нормально работает лишь в \XeLaTeX.

% \ifdefined\NoMinted
% Но к сожалению, у вас, по-видимому, не установлен Python или pygmentize.
% \else
% Можно пользоваться расширенным BFN:

% \begin{listing}[H]
% \begin{ebnfcode}
%  letter = "A" | "B" | "C" | "D" | "E" | "F" | "G"
%        | "H" | "I" | "J" | "K" | "L" | "M" | "N"
%        | "O" | "P" | "Q" | "R" | "S" | "T" | "U"
%        | "V" | "W" | "X" | "Y" | "Z" ;
% digit = "0" | "1" | "2" | "3" | "4" | "5" | "6" | "7" | "8" | "9" ;
% symbol = "[" | "]" | "{" | "}" | "(" | ")" | "<" | ">"
%        | "'" | '"' | "=" | "|" | "." | "," | ";" ;
% character = letter | digit | symbol | "_" ;
 
% identifier = letter , { letter | digit | "_" } ;
% terminal = "'" , character , { character } , "'" 
%          | '"' , character , { character } , '"' ;
 
% lhs = identifier ;
% rhs = identifier
%      | terminal
%      | "[" , rhs , "]"
%      | "{" , rhs , "}"
%      | "(" , rhs , ")"
%      | rhs , "|" , rhs
%      | rhs , "," , rhs ;
 
% rule = lhs , "=" , rhs , ";" ;
% grammar = { rule } ;
% \end{ebnfcode}
% \caption{EBNF определённый через EBNF}
% \label{lst:ebnf}
% \end{listing}

% А вот в листинге \ref{lst:c} на языке C работают русские комменты. Спасибо Pygments и Minted за это.

% \begin{listing}[H]
% \cfile{inc/src/test.c}
% \caption{Пример — test.c} 
% \end{listing}
% \label{lst:c}

% \fi
% \fi
% % Для вставки реального кода лучше использовать \texttt{\textbackslash lstinputlisting} (который понимает
% % UTF8) и стили \Code{realcode} либо \Code{simplecode} (в зависимости от размера куска).




% Можно также использовать окружение \Code{verbatim}, если \Code{listings} чем-то не
% устраивает. Только следует помнить, что табы в нём <<съедаются>>. Существует так же команда \Code{\textbackslash{}verbatiminput} для вставки файла.

% \begin{verbatim}
% a_b = a + b; // русский комментарий
% if (a_b > 0)
%     a_b = 0;
% \end{verbatim}

%%% Local Variables:
%%% mode: latex
%%% TeX-master: "rpz"
%%% End:

\chapter{Экспериментальный раздел}
\label{cha:research}


\section{Технические характеристики}
Тестирование выполнялось на устройстве \cite{lenovo} со следующими техническими характеристиками:
\begin{itemize}
	\item Операционная система Ubuntu 20.04.03 LTS;
	\item Память 16 GiB (4,5 GiB выделено для нужд графического ядра)
	\item Процессор AMD® Ryzen 5 5500u --- 12 потоков
    \item Видеопроцессор AMD® Radeon RX Vega 7 --- 448 потоков
\end{itemize}

\section{Время выполнения алгоритмов}

Для замеров времени использовалась стандартного класса языка С++ \verb|std::chrono::high_resolution_clock|. Данный класс предоставляет интерфейс замеров времени с максимальной возможной точностью.

\subsection{Анализ времени работы этапов построения изображения}

Был произведен замер времени генерации изображения, разбитый на логические этапы. При тестировании на сцене (представлена на рисунке \ref{fig:scene_exp}) были видимы 7 моделей, разрешение генерируемого изображения 1848x1016 пикселей, 20 семплов методов размытия и размер участка изображения 40x40 пикселей. В данной сцене камера следует за движущемся автомобилем. Результаты замеров приведены в таблице \ref{tbl:time}. 


\begin{figure}[h]
    \centering
    \includegraphics[width=0.8\linewidth]{img/exp/s_1.png} 

    \caption{Сцена, используемая для замеров времени}
    \label{fig:scene_exp}
\end{figure} 


\begin{longtable}{|l|l|}

    \caption{Результаты замеров времени работы этапов построения изображения}
    \label{tbl:time}
    \\
    \hline
        Этап построения                                     & Время, мс \\ \hline    \hline
        буферов кадра, глубины и скорости пикселей          & 14.964   \\ \hline
        буфера максимальной скорости участка                & 0.640    \\ \hline
        размытия с помощью скорости пикселя                 & 1.429    \\ \hline
        размытия с помощью максимальной скорости участка    & 6.204    \\ \hline
\end{longtable}

Из результатов замеров можно сделать вывод, что самым долгим этапом является формирование буферов кадра, глубины и скорости пикселей. Данный этап в 2.2 раза дольше этапов построения размытия с помощью максимальной скорости участка и буфера максимальной скорости участка. И в 10.4 раза больше этапа размытия с помощью скорости пикселя. Из данного факта можно сделать вывод, что методы размытия изображения могут работать в режиме реального времени, но вклад во время формирования итогового изображения зависит от выбранного алгоритма размытия. Данный вопрос будет рассмотрен далее.


\subsection{Анализ времени работы методов размытия от количества семплов}

Было произведено исследование зависимости времени работы методов размытия от количества семплов. Результаты представлены на рисунках \ref{fig:plot_time} и \ref{fig:plot_fps}

\begin{figure}[H]
    \centering
    
    \begin{tikzpicture}
        \begin{axis} [
            legend pos = north west, 
            % ymin = 0, 
            height=0.5\textwidth,
            grid = major,
            xlabel={Количество семплов},
            ylabel={Количество миллисекунд},
            table/col sep = semicolon,
            /pgf/number format/1000 sep={}
        ]
        \legend{ 
            C помощью скорости пикселя,
            C помощью максимальной скорости участка
        };
        \addplot [no markers, thick, blue] table [x=x, y=y] {graphs/samples_pixel.csv};
        \addplot [no markers, thick, red] table [x=x, y=y] {graphs/samples_neighbor.csv};
        % \addplot table [x=x, y=y] {graphs/samples_aprox.csv};
        \end{axis}
    \end{tikzpicture}

    \caption{Зависимость времени генерации итогового изображения для реализаций методов размытия от количества семплов}
    \label{fig:plot_time}
\end{figure} 

\begin{figure}[H]
    \centering
    
    \begin{tikzpicture}
        \begin{axis} [
            legend pos = north east, 
            % ymin = 0, 
            grid = major,
            xlabel={Количество семплов},
            ylabel={Количество кадров в секунду},
            height=0.5\textwidth,
            table/col sep = semicolon,
            /pgf/number format/1000 sep={}
        ]
        \legend{ 
            C помощью скорости пикселя,
            C помощью максимальной скорости участка 
        };
        \addplot [no markers, thick, blue] table [x=x, y=y] {graphs/fps_pixel.csv};
        \addplot [no markers, thick, red] table [x=x, y=y] {graphs/fps_neighbor.csv};
        \end{axis}
    \end{tikzpicture}

    \caption{Зависимость частоты генерации итогового изображения для методов размытия от количества семплов}
    \label{fig:plot_fps}
\end{figure} 

В результате исследования было установлено, что метод размытия с помощью скорости пикселя работает быстрее для исследуемого диапазона количества семплов. Метод размытия с помощью скорости пикселя быстрее метода размытия с помощью максимальной скорости участка и глубины в 1.4 раза при генерации смаза с помощью 40 семплов. Было установлено, что метод размытия с помощью скорости пикселя может работать с частотой обновления кадров 60 Гц при 31 семпле, а метод размытия с помощью максимальной скорости участка и глубины --- при 16 семплах. 

\section{Визуальные характеристики методов размытия}

\subsection{Анализ визуальных характеристик методов размытия от количества семплов}

Были сформированы изображения размытия с помощью скорости участка для разного количества семплов. Они представлены на рисунке \ref{fig:result_samples_neighbor}.

\begin{figure}[h]
    \centering
    \begin{minipage}[h]{0.45\linewidth}
        \center{\includegraphics[width=0.99\linewidth]{img/exp/cropped/s_2.png} \\ a)}
    \end{minipage}
    \hfill
    \begin{minipage}[h]{0.45\linewidth}
        \center{\includegraphics[width=0.99\linewidth]{img/exp/cropped/s_5.png} \\ б)}
    \end{minipage}
    

    \begin{minipage}[h]{0.45\linewidth}
        \center{\includegraphics[width=0.99\linewidth]{img/exp/cropped/s_10.png} \\ в)}
    \end{minipage}
    \hfill
    \begin{minipage}[h]{0.45\linewidth}
        \center{\includegraphics[width=0.99\linewidth]{img/exp/cropped/s_20.png} \\ г)}
    \end{minipage}

    \begin{minipage}[h]{0.45\linewidth}
        \center{\includegraphics[width=0.99\linewidth]{img/exp/cropped/s_40.png} \\ д)}
    \end{minipage}
    \hfill
    \begin{minipage}[h]{0.45\linewidth}
        \center{\includegraphics[width=0.99\linewidth]{img/exp/cropped/s_100.png} \\ е)}
    \end{minipage}


    \caption{Результат размытия с помощью скорости участка для \\ а) 2 семплов б) 5 семплов  в) 10 семплов  г) 20 семплов  д) 40 семплов  e) 100 семплов }
    \label{fig:result_samples_neighbor}
\end{figure} 



На изображениях c количеством семплов менее 20 видны недостатки размытия, проявляющиеся в виде шумного смаза. При 20 семплах шум проявляется в меньшей степени. При более 40 семплах результат работы метода не имеет визуальных различий. Из данных утверждений следует, что для изображения среднего качества можно использовать как минимум 20 семплов, а для достижения реалистичности более 40 семплов.

Были сформированы изображения размытия с помощью скорости пикселя для разного количества семплов. Они представлены на рисунке \ref{fig:result_samples_pixel}.


\begin{figure}[H]
    \centering
    \begin{minipage}[h]{0.45\linewidth}
        \center{\includegraphics[width=0.99\linewidth]{img/exp/cropped/ss_20.png} \\ a)}
    \end{minipage}
    \hfill
    \begin{minipage}[h]{0.45\linewidth}
        \center{\includegraphics[width=0.99\linewidth]{img/exp/cropped/ss_40.png} \\ б)}
    \end{minipage}

    \begin{minipage}[h]{0.45\linewidth}
        \center{\includegraphics[width=0.99\linewidth]{img/exp/cropped/ss_100.png} \\ в)}
    \end{minipage}
    


    \caption{Результат размытия с помощью скорости пикселя для \\ а) 20 семплов б) 40 семплов  в) 100 семплов}
    \label{fig:result_samples_pixel}
\end{figure} 

На изображениях c количеством семплов менее 40 видны недостатки размытия, проявляющиеся в виде разрывов смаза. При 40 семплах разрывы проявляется в меньшей степени. При более 40 семплах результат работы не имеет разрывов смаза. Из данных утверждений следует, что для изображения среднего качества можно использовать как минимум 40 семплов.


Из данного анализа, можно сделать вывод, что размытие с помощью скорости участка обладает лучшими визуальными характеристиками при равном количестве семплов. Для достижения наилучших визуальных характеристик каждого метода необходимо использовать более 40 семплов.  

\subsection{Анализ визуальных недостатков методов размытия}

В результате исследования были выявлены два визуальных недостатка исследуемых методов размытия. 

Метод размытия с помощью скорости пикселя размывает статичные объекты в системе координат камеры, если задний план находится в движении, а передний статичен.  Метод размытия с помощью скорости участка и буфера глубины лишен данного недостатка в силу учета глубины изображения. Данный визуальный недостаток метода представлен на рисунке \ref{fig:exp_art_1}.


\begin{figure}[H]
    \centering
    \begin{minipage}[h]{0.45\linewidth}
        \center{\includegraphics[width=0.99\linewidth]{img/art/cropped/1_pixel.png} \\ a)}
    \end{minipage}
    \hfill
    \begin{minipage}[h]{0.45\linewidth}
        \center{\includegraphics[width=0.99\linewidth]{img/art/cropped/1_neighbor.png} \\ б)}
    \end{minipage}
    

    \caption{Результат размытия с помощью  \\ а) скорости пикселя б) скорости участка и буфера глубины}
    \label{fig:exp_art_1}
\end{figure} 


Другой визуальный недостаток, который был обнаружен, проявляется в потере части смаза из-за перекрытия неподвижным передним планом движущегося объекта и движущегося в том же направлении заднего плана. Данный визуальный недостаток проявляется в двух исследуемых методах. В силу предыдущего визуального недостатка метод размытия с помощью скорости пикселя размывает изображение, используя изображение переднего плана, а метод размытия с помощью скорости участка и буфера глубины не учитывает передний план, но не обладает изображением изображением движущегося объекта из-за чего происходит потеря части смаза. Данный визуальный недостаток методов представлен на рисунке \ref{fig:exp_art_2}.

\begin{figure}[H]
    \centering
    \begin{minipage}[h]{0.45\linewidth}
        \center{\includegraphics[width=0.99\linewidth]{img/art/cropped/2_pixel.png} \\ a)}
    \end{minipage}
    \hfill
    \begin{minipage}[h]{0.45\linewidth}
        \center{\includegraphics[width=0.99\linewidth]{img/art/cropped/2_neighbor.png} \\ б)}
    \end{minipage}
    

    \caption{Результат размытия с помощью  \\ а) скорости пикселя б) скорости участка и буфера глубины}
    \label{fig:exp_art_2}
\end{figure} 


Необходимо заметить, что данный визуальный недостаток при генерации анимации будет проявляться на малом количестве кадров, что делает его мало заметным при воспроизведении анимации, так как недостаток будет проявляться в течении промежутка времени до 100 мс. Но при генерации одиночных кадров данными методами визуальный недостаток будет проявляться.  


\section{Вывод}


Метод размытия с помощью скорости пикселя может быть использован в режиме реального времени при генерации размытия движения камеры с количеством семплов более 40. При генерации размытия перекрывающихся движущихся объектов могут возникать графические недостатки, которые будут заметны во время воспроизведения анимации. 


Метод размытия с помощью скорости участка и буфера глубины  может быть использован в режиме реального времени при генерации размытия движения камеры и других объектов с количеством семплов не более 40. При этом визуальные недостатки проявляться не будут в течении промежутка времени большего 100 мс.



%%% Local Variables:
%%% mode: latex
%%% TeX-master: "rpz"
%%% End:


% % В данном разделе проводятся вычислительные эксперименты.
% % А на рис.~\ref{fig:spire01} показана схема мыслительного процесса автора...

% % \begin{figure}
% %   \centering
% %   \includegraphics[width=\textwidth]{inc/svg/pic01}
% %   \caption{Как страшно жить}
% %   \label{fig:spire01}
% % \end{figure}


%%% Local Variables:
%%% mode: latex
%%% TeX-master: "rpz"
%%% End:

% \include{60-economics}
% \include{70-bzd}

\backmatter %% Здесь заканчивается нумерованная часть документа и начинаются ссылки и

\Conclusion % заключение к отчёту

В результате проделанной работы были решены поставленные задачи. Были определены требования для генерации смаза движения. На основании данных требований были проанализированы методы генерации смаза изображения. Был выбран и модифицирован метод построения объемного изображения полигональных моделей с учетом требований  методов генерации размытия. Была спроектирована архитектура ПО, позволяющая вносить в него модификацию для дальнейших исследований. Были реализованы методы размытия изображения и были проанализированы временные и визуальные характеристики методов размытия. В результате анализа были предложены области применения отдельных методов. 

Цель работы достигнута. Было разработано ПО для генерации эффекта размытия движения полигональных моделей в режиме реального времени.

%%% Local Variables: 
%%% mode: latex
%%% TeX-master: "rpz"
%%% End: 
%% заключение


\include{81-biblio}


\appendix   % Тут идут приложения



\begin{landscape}
	\chapter{Диаграмма классов}
	\label{cha:appendix1}

	\begin{figure}
		\centering
		\includesvg[width=1\columnwidth,inkscapelatex=false]{img/umls/svg/Model!Overview_4.svg}
		\caption{Диаграмма общего представленияы классов}
	\end{figure}

\end{landscape}

\begin{landscape}
	\begin{figure}
		\centering
		\includesvg[width=1\columnwidth,inkscapelatex=false]{img/umls/svg/Model!Main_0.svg}
		\caption{Развернутая диаграмма классов}
	\end{figure}
\end{landscape}


%%% Local Variables: 
%%% mode: latex
%%% TeX-master: "rpz"
%%% End: 


% \include{91-appendix2}

\end{document}

%%% Local Variables:
%%% mode: latex
%%% TeX-master: t
%%% End:
