\Conclusion 

В данной работе был рассмотрен вопрос размытия движения объектов. Проанализировав природу появления смаза, были выдвинуты требования для генерации смаза движения. На основание данных требований были рассмотрены алгоритмы генерации размытия движения. В результате анализа были рассмотрены разные сферы применения тех или иных смазов.

Были проанализированы алгоритмы удаления невидимых линий и поверхностей. Были приведены выкладки по модификации алгоритма удаления невидимых линий с помощью z-буфера с целью формирования дополнительного буфера скоростей пикселей. Были приведены выкладки по реализации методов смаза в виде математических выражений и псевдокода.

Была разработана гибкая архитектура ПО, позволяющая легко добавлять новые типы объектов сцены, методы размытия движения, функционал объектов сцены, заменять движок отрисовки и графический интерфейс пользователя. 

Были выдвинуты требуемые инструменты для разработки ПО и его тестирования. Был спроектирован графический интерфейс пользователя приложения.  

Итого, в результате проделанной работы было спроектировано ПО для генерации эффекта размытия движения полигональных моделей и даны рекомендации по его программной реализации.

%%% Local Variables: 
%%% mode: latex
%%% TeX-master: "rpz"
%%% End: 
