\thispagestyle{empty}
\begin{center}
    \fontsize{12pt}{0.3\baselineskip}\selectfont \textbf{Министерство науки и высшего образования Российской Федерации \\ Федеральное государственное бюджетное образовательное учреждение \\ высшего образования \\ <<Московский государственный технический университет имени Н.Э. Баумана \\ (национальный исследовательский университет)>> \\ (МГТУ им. Н.Э. Баумана)}

    \fontsize{11pt}{0.3\baselineskip}\selectfont
    \noindent \makebox[\linewidth]{\rule{\textwidth}{4pt}} \makebox[\linewidth]{\rule{\textwidth}{1pt}}
\end{center}
\begin{flushright}
    \fontsize{11pt}{0.5\baselineskip}\selectfont УТВЕРЖДАЮ \\ Заведующий кафедрой \textbf{ИУ7}, \\ \textbf{\hspace*{2.5cm}} \uline{\hspace*{2cm}} \textbf{И.В. Рудаков} \\ <<\uline{\hspace*{1cm}}>> \uline{\hspace*{2.5cm}} \textbf{2021} г.
\end{flushright}

\vfill

\begin{center}
    \fontsize{18pt}{0.7\baselineskip}\selectfont \textbf{З А Д А Н И Е}\\
    \fontsize{16pt}{\baselineskip}\selectfont \textbf{на выполнение курсовой работы}
\end{center}

\normalsize
\begin{flushleft}
    \fontsize{11pt}{0.5\baselineskip}\selectfont
    \setlength{\parskip}{0.1em}
    по дисциплине \uline{\hfill Компьютерная графика \hfill}

    Студент группы \uline{\hfill Ларин Владимир Николаевич \hfill}

    Тема курсовой работы \uline{\hfill ТЕМА \hfill}

    Направленность КР (учебная, исследовательская, практическая, производственная, др.) \\
    \uline{\hfill ТЕМА \hfill}

    Источник тематики (кафедра, предприятие, НИР) \uline{\hfill кафедра \hfill}

    График выполнения работы:  25\% к \uline{?} нед., 50\% к \uline{?} нед., 75\% к \uline{?} нед., 100\% к \uline{?} нед.

    \textbf{\textit{Задание}}
    \uline{\hfill ? \hfill}

    \uline{\hfill ? \hfill}

    \uline{\hfill ? \hfill}

    \uline{\hfill ? \hfill}

    \textbf{\textit{Оформление курсовой работы:}}

    Расчетно-пояснительная записка на 25-30  листах формата А4.
    
    \uline{
        Расчетно-пояснительная записка должна содержать постановку введение, аналитическую часть, конструкторскую часть, технологическую часть, экспериментально-исследовательский раздел, заключение, список литературы, приложения.
    }


    Перечень графического материала (плакаты, схемы, чертежи и т.п.)

    \uline{На защиту проекта должна быть представлена презентация, состоящая из 15-20 слайдов. На слайдах должны быть отражены: постановка задачи, использованные методы и алгоритмы, расчетные соотношения, структура комплекса программ, диаграмма классов, интерфейс, характеристики разработанного ПО, результаты проведенных исследований.
    }

    Дата выдачи задания <<\uline{\hspace*{5mm}}>> \uline{\hspace*{2.5cm}} 20\uline{\hspace*{5mm}} г.
\end{flushleft}


\vfill

\begin{table}[h!]
    \fontsize{11pt}{0.7\baselineskip}\selectfont
    \centering
    \begin{signstabular}[0.7]{p{8cm} >{\centering\arraybackslash}p{3.8cm} >{\centering\arraybackslash}p{3.8cm}}
        \textbf{Руководитель курсовой работы} & \uline{\hspace*{3.8cm}} & \uline{\hfill А.А. Волкова \hfill} \\
        \rule{0pt}{0pt} & \fontsize{9pt}{\baselineskip}\selectfont (Подпись, дата) & \fontsize{9pt}{\baselineskip}\selectfont (И.О. Фамилия)
    \end{signstabular}
    \begin{signstabular}[0.7]{p{8cm} >{\centering\arraybackslash} >{\centering \arraybackslash}p{3.8cm} >{\centering\arraybackslash}p{3.8cm}}
        \textbf{Студент} & \uline{\hspace*{3.8cm}} & \uline{\hfill В.Н. Ларин \hfill} \\
        \rule{0pt}{0pt} & \fontsize{9pt}{\baselineskip}\selectfont (Подпись, дата) & \fontsize{9pt}{\baselineskip}\selectfont (И.О. Фамилия)

    \end{signstabular}
\end{table}
\clearpage

\begin{flushleft}
    \fontsize{11pt}{0.5\baselineskip}\selectfont
    \uline{Примечание:} Задание оформляется в двух экземплярах -- один выдается студенту, второй хранится на кафедре
\end{flushleft}