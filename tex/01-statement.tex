\thispagestyle{empty}
\begin{center}
    \fontsize{11pt}{0.3\baselineskip}\selectfont \textbf{Министерство науки и высшего образования Российской Федерации \\ Федеральное государственное бюджетное образовательное учреждение \\ высшего образования \\ <<Московский государственный технический университет имени Н.Э. Баумана \\ (национальный исследовательский университет)>> \\ (МГТУ им. Н.Э. Баумана)}
    % \fontsize{1pt}{0.3\baselineskip}\selectfont
    \makebox[\linewidth]{\rule{\textwidth}{3pt}}
    \makebox[\linewidth]{\rule{\textwidth}{3pt}}
    
    \vspace{\baselineskip}

    \fontsize{12pt}{\baselineskip}\selectfont 

    Кафедра << \uline{Программное обеспечение ЭВМ и информационные технологии} >> ( \uline{ИУ7} )
\end{center}


\begin{center}
    \fontsize{18pt}{\baselineskip}\selectfont \textbf{З А Д А Н И Е}\\
    \fontsize{16pt}{\baselineskip}\selectfont \textbf{на прохождение производственной практики}
\end{center}

\normalsize

\begingroup
\fontsize{12pt}{1\baselineskip}\selectfont
% \setlength{\parskip}{0.1em}
\setlength{\parindent}{0em}
на предприятии \uline{\hfill МГТУ им. Н. Э. Баумана, каф. ИУ7 \hfill}

Студент \uline{\hfill Ларин Владимир Николаевич ИУ7-44Б \hfill}

Во время прохождения производственной практики студент должен:

1. Спроектировать программу синтеза изображения полигональных моделей с возможностью генерации эффекта размытия их движения.  

2. Предоставить пользователю возможность загружать полигональные модели из файла. 

3. Предоставить возможность пользователю устанавливать заданные параметры движения и размытия.


\vfill

Дата выдачи задания
 <<\uline{\mbox{\hspace*{5mm}}}>> \uline{\mbox{\hspace*{2.5cm}}} 20\uline{21} г.

\endgroup

\vfill

\begin{table}[h!]
    \fontsize{12pt}{0.7\baselineskip}\selectfont
    \centering
 

    \begin{signstabular}[0.7]{p{7.25cm} >{\centering\arraybackslash}p{4cm} >{\centering\arraybackslash}p{4cm}}
        Руководитель практики от кафедры & \uline{\mbox{\hspace*{4cm}}} & \uline{\hfill Куров А.В. \hfill} \\
        & \scriptsize (Подпись, дата) & \scriptsize (Фамилия И.О.)
    \end{signstabular}
    \vspace{\baselineskip}

    \begin{signstabular}[0.7]{p{7.25cm} >{\centering\arraybackslash}p{4cm} >{\centering\arraybackslash}p{4cm}}
        Студент & \uline{\mbox{\hspace*{4cm}}} & \uline{\hfill Ларин В.Н. \hfill} \\
        & \scriptsize (Подпись, дата) & \scriptsize (Фамилия И.О.)
    \end{signstabular}

    \vspace{\baselineskip}
\end{table}


% \begin{flushleft}
%     \fontsize{11pt}{0.5\baselineskip}\selectfont
%     \uline{Примечание:} Задание оформляется в двух экземплярах -- один выдается студенту, второй хранится на кафедре
% \end{flushleft}

\clearpage
\thispagestyle{empty}

\begin{center}
    \fontsize{12pt}{\baselineskip}\selectfont
    \textit{Дополнительные указания по проектированию}
\end{center}

\begingroup
\fontsize{12pt}{0.7\baselineskip}\selectfont
\setlength{\parskip}{0em}
\setlength{\parindent}{0em}

\uline{\mbox{\hspace*{1.25cm}} Для синтеза изображения полигональной модели должна использоваться однотонная закраска. Пользователь должен иметь возможность выбрать цвет модели из заданных. Модель должна состоять только из полигонов выпуклой формы. На сцене может присутствовать до 5 моделей. Суммарное количество полигонов всех моделей не должно превышать 5 000 штук. На сцене должна присутствовать одна камера. Камера и модели являются объектами сцены. Пользователь должен иметь возможность настраивать следующие параметры объектов: начальное и конечное положения в пространстве, начальный и конечный повороты, начальный и конечный масштабы. Данные параметры должны устанавливаться для осей $x$, $y$, $z$ относительно центра объекта.
    \hfill
}


\uline{\mbox{\hspace*{1.25cm}} Пользователь может задавать время начала анимации и ее продолжительность для каждого объекта. Продолжительность анимации должна не превышать 60 секунд. Время начала анимации должно быть больше 0 с и меньше 60 с. Движение должно происходить по линейному закону.\hfill}


\uline{\mbox{\hspace*{1.25cm}} Пользователь должен иметь возможность воспроизвести анимацию объектов с произвольного момента времени из отрезка [0; 120] c  и прекратить воспроизведение анимации по нажатию на кнопку останова анимации. После достижения точки времени 120 с воспроизведение должно прекратиться. Пользователь должен иметь возможность просмотреть сгенерированное изображение для произвольной точки времени с точностью до 0,1 с из отрезка [0; 120] c \hfill}


\endgroup
\normalsize
