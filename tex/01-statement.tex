\thispagestyle{empty}
\begin{center}
    \fontsize{11pt}{0.3\baselineskip}\selectfont \textbf{Министерство науки и высшего образования Российской Федерации \\ Федеральное государственное бюджетное образовательное учреждение \\ высшего образования \\ <<Московский государственный технический университет имени Н.Э. Баумана \\ (национальный исследовательский университет)>> \\ (МГТУ им. Н.Э. Баумана)}
    % \fontsize{1pt}{0.3\baselineskip}\selectfont
    \makebox[\linewidth]{\rule{\textwidth}{3pt}}
    \begin{flushright}
        \fontsize{11pt}{0.5\baselineskip}\selectfont
            УТВЕРЖДАЮ \\ Заведующий кафедрой \textbf{ИУ7}, \\
            \uline{\mbox{\hspace*{2cm}}}
            \textbf{И.В. Рудаков} \\
            <<\uline{\mbox{\hspace*{1cm}}}>>
            \uline{\mbox{\hspace*{2.5cm}}}
            \textbf{2021} г.
    \end{flushright}
\end{center}


\begin{center}
    \fontsize{18pt}{0.6\baselineskip}\selectfont \textbf{З А Д А Н И Е}\\
    \fontsize{16pt}{0.6\baselineskip}\selectfont \textbf{на выполнение курсового проекта}
\end{center}

\normalsize

\begingroup
\fontsize{12pt}{0.5\baselineskip}\selectfont
\setlength{\parskip}{0.1em}
\setlength{\parindent}{0em}
по дисциплине \uline{\hfill Компьютерная графика \hfill}

Студент группы \uline{\hfill ИУ7-54Б \hfill}

\uline{\hfill Ларин Владимир Николаевич \hfill}

Тема курсовой работы \uline{Разработка   программного   обеспечения   для генерации размытия движения полигональных моделей \hfill}

Направленность КР (учебная, исследовательская, практическая, производственная, др.)

\uline{\hfill учебная \hfill}

Источник тематики (кафедра, предприятие, НИР) \uline{\hfill кафедра \hfill}

График выполнения работы:  25\% к \uline{4} нед., 50\% к \uline{7} нед., 75\% к \uline{11} нед., 100\% к \uline{14} нед.

\textbf{\textit{Задание}}
\uline{Разработать программу синтеза изображения полигинальных моделей с возможностью генерации эффекта размытия их движения.  Предоставить пользователю возможность загружать полигональные модели из файла. Предоставить возможность пользователю устанавливать заданные параметры движения и размытия. Исследовать зависимость скорости синтеза размытия изображения от характеристик алгоритма.
    \hfill}

\textbf{\textit{Оформление курсовой работы:}}

Расчетно-пояснительная записка на 25-30  листах формата А4.

\uline{Расчетно-пояснительная записка должна содержать постановку введение, аналитическую часть, конструкторскую часть, технологическую часть, экспериментально-исследовательский раздел, заключение, список литературы, приложения.
    \hfill}

Перечень графического материала (плакаты, схемы, чертежи и т.п.)

\uline{На защиту проекта должна быть представлена презентация, состоящая из 15-20 слайдов. На слайдах должны быть отражены: постановка задачи, использованные методы и алгоритмы, расчетные соотношения, структура комплекса программ, диаграмма классов, интерфейс, характеристики разработанного ПО, результаты проведенных исследований.
    \hfill}

Дата выдачи задания
 <<\uline{\mbox{\hspace*{5mm}}}>> \uline{\mbox{\hspace*{2.5cm}}} 20\uline{21} г.

\endgroup

\vfill

\begin{table}[h!]
    \fontsize{12pt}{0.7\baselineskip}\selectfont
    \centering
    \begin{signstabular}[0.7]{p{7.25cm} >{\centering\arraybackslash}p{4cm} >{\centering\arraybackslash}p{4cm}}
        Студент группы ИУ7-54Б & \uline{\mbox{\hspace*{4cm}}} & \uline{\hfill В.Н. Ларин  \hfill} \\
        & \scriptsize (Подпись, дата) & \scriptsize (И.О. Фамилия)
    \end{signstabular}

    \vspace{\baselineskip}

    \begin{signstabular}[0.7]{p{7.25cm} >{\centering\arraybackslash}p{4cm} >{\centering\arraybackslash}p{4cm}}
        Руководитель курсового проекта & \uline{\mbox{\hspace*{4cm}}} & \uline{\hfill А.А. Волкова \hfill} \\
        & \scriptsize (Подпись, дата) & \scriptsize (И.О. Фамилия)
    \end{signstabular}
    \vspace{\baselineskip}
\end{table}


% \begin{flushleft}
%     \fontsize{11pt}{0.5\baselineskip}\selectfont
%     \uline{Примечание:} Задание оформляется в двух экземплярах -- один выдается студенту, второй хранится на кафедре
% \end{flushleft}

\clearpage
\newgeometry{left=20mm,right=30mm,top=20mm,bottom=20mm}
\thispagestyle{empty}

\begin{center}
    \fontsize{12pt}{\baselineskip}\selectfont
    \textit{Дополнительные указания по проектированию}
\end{center}

\begingroup
\fontsize{12pt}{0.7\baselineskip}\selectfont
\setlength{\parskip}{0em}
\setlength{\parindent}{0em}

\uline{\mbox{\hspace*{1.25cm}} Для синтеза изображения полигональной модели должна использоваться однотонная закраска. Пользователь должен иметь возможность выбрать цвет модели из заданных. Модель должна состоять только из полигонов выпуклой формы. На сцене может присутствовать до 5 моделей. Сумарное количество полигонов всех моделей не должно превышать 5 000 штук. На сцене должна присутствовать одна камера. Камера и модели являются объектами сцены. Пользователь должен иметь возможность настраивать следующие параметры объектов: начальное и конечное положения в пространстве, начальный и конечный повороты, начальный и конечный масштабы. Данные параметры должны устанавливаться для осей $x$, $y$, $z$ относительно центра объекта.
    \hfill
}


\uline{\mbox{\hspace*{1.25cm}} Пользователь может задавать время начала анимации и ее продолжительность для каждого объекта. Продолжительность анимации должна не превышать 60 секунд. Время начала анимации должно быть больше 0 с и меньше 60 с. Движение должно происходить по линейному закону.\hfill}


\uline{\mbox{\hspace*{1.25cm}} Пользователь должен иметь возможность воспроизвести анимацию объектов с произвольного момента времени из отрезка [0; 120] c  и прекратить воспроизведение анимации по нажатию на кнопку останова анимации. После достижения точки времени 120 с воспроизведение должно прекратиться. Пользователь должен иметь возможность просмотреть сгенерированное изображение для произвольной точки времени с точностью до 0,1 с из отрезка [0; 120] c \hfill}


\endgroup
\normalsize
\restoregeometry
